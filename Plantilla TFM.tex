%%%%%%%%%%%%%%%%%%%%%%%%%%%%%%%%%%%%%%%%%%%%%%%%%%%%%%%%
%%% Plantilla desarrollada por Jorge Carrillo de Albornoz
%%% para los trabajos de Fin de Master del M�ster en
%%% Lenguajes y Sistemas Inform�ticos de la UNED
%%% Cualquier duda o comentario: jcalbornoz@lsi.uned.es
%%%%%%%%%%%%%%%%%%%%%%%%%%%%%%%%%%%%%%%%%%%%%%%%%%%%%%%%

\documentclass[a4paper,11pt,twoside,openright,spanish]{book}

\usepackage{latexsym}
\usepackage[latin1]{inputenc}
\usepackage[activeacute,spanish]{babel}
\frenchspacing
\usepackage{eucal}
\usepackage{multirow}
\usepackage{fancyhdr}
\usepackage[final,pdftex]{graphicx}
\usepackage{subfigure}
\usepackage{amsmath}
\usepackage{fullname_esp}
\usepackage[colorlinks=true, linkcolor=blue]{hyperref}
\usepackage{pdfpages}
\usepackage{algorithmic}
\usepackage{algorithm}
\usepackage{listings}
\lstset{
	basicstyle=\ttfamily,
	frame=single
}
\usepackage{color}
\usepackage{setspace}
\usepackage{caption}
\usepackage{booktabs}




%%%%%%%%%%%%%%%%%%%%%%%%%%%%%%%%%%%%%%%%%%%%%%%%%%%%%%%%
%%% SCRIPT DE TODO'S O TAREAS PENDIENTES CON BOCADILLOS.
%%%%%%%%%%%%%%%%%%%%%%%%%%%%%%%%%%%%%%%%%%%%%%%%%%%%%%%%
\usepackage{tikz}
\makeatletter \newcommand \listoftodos{\section*{Todo list} \@starttoc{tdo}}
\newcommand\l@todo[2]
    {\par\noindent \textit{#2}, \parbox{10cm}{#1}\par} \makeatother

\definecolor{orange}{rgb}{1,0.5,0}
\tikzstyle{notestyle} += [text width=\marginparwidth]
\tikzstyle{notestyleleft} += [text width=0.9\marginparwidth]
\tikzstyle{notestyle} = [draw=black, fill=orange, text width = 3cm]
\tikzstyle{notestyleleft} = [notestyle]
\tikzstyle{connectstyle} = [draw = orange, thick]
% Command for inserting a todo item
\newcommand{\todo}[1]{%
% Add to todo list
\addcontentsline{tdo}{todo}{\protect{#1}}%
%
\begin{tikzpicture}[remember picture, baseline=-0.75ex]%
    \node [coordinate] (inText) {};
\end{tikzpicture}%
%
% Make the margin par
\marginpar[%
{% Draw note in left margin
    \tikz[remember picture] \draw node[notestyleleft] (inNote) {#1};%
    \begin{tikzpicture}[remember picture, overlay]%
        \draw[connectstyle]
            ([yshift=-0.2cm] inText)
                -| ([xshift=0.2cm] inNote.east)
                -| (inNote.east);
    \end{tikzpicture}%
}%
]{% Draw note in right margin
    \tikz[remember picture] \draw node[notestyle] (inNote) {#1};%
    \begin{tikzpicture}[remember picture, overlay]%
        \draw[connectstyle]
            ([yshift=-0.2cm] inText)
                -| ([xshift=-0.2cm] inNote.west)
                -| (inNote.west);
    \end{tikzpicture}%
}%
}%

%%%%%%%%%%%%%%%%%%%%%%%%%%%%%%%%%%
%%%%%%%%%%%%%%%%%%%%%%%%%%%%%%%%%%

\floatname{algorithm}{Algoritmo}
\pagestyle{fancy}
\renewcommand{\chaptermark}[1]{\markboth{\textsc{Cap�tulo} \thechapter. #1}{}}
\renewcommand{\sectionmark}[1]{\markright{\thesection\ #1}}

\fancyhf{}
\fancyhead[LE,RO]{\thepage}
\fancyhead[LO]{\rightmark}
\fancyhead[RE]{\leftmark}

%% Dejar espacio para la regla
\renewcommand{\headrulewidth}{0.02cm} %Son 0.5pt

% Redefinici�n de 'plain' para la primera p�gina del cap�tulo
\fancypagestyle{plain}{
\fancyhf{}
\renewcommand{\headrulewidth}{0pt}
\renewcommand{\footrulewidth}{0pt}
}

% Limpiar las cabeceras de las p�ginas que quedan completamente blancas
\makeatletter
  \def\cleardoublepage{\clearpage\if@twoside \ifodd\c@page\else
  \vspace*{\fill}
    \thispagestyle{empty}
    \newpage
    \if@twocolumn\hbox{}\newpage\fi\fi\fi}
\makeatother


\setlength\headheight{20pt}


%%%%%%%%%%%%%%%%%%%%%%%%%%%%%%%%%%%%%%%%%%%%%%
%%% COMIENZA EL DOCUMENTO
%%%%%%%%%%%%%%%%%%%%%%%%%%%%%%%%%%%%%%%%%%%%%%

\pagenumbering{roman}
\makeindex
\begin{document}

\newlength{\centeroffset}
\setlength{\centeroffset}{-0.5\oddsidemargin}
\addtolength{\centeroffset}{0.5\evensidemargin}
\thispagestyle{empty}
\vspace*{\stretch{1}}
\setstretch{1.2}

\noindent
\hspace*{\centeroffset}

\renewcommand\contentsname{�ndice general}
\renewcommand\listfigurename{�ndice de Figuras}
\renewcommand\listtablename{�ndice de Tablas}
\renewcommand\bibname{Referencias}
\renewcommand\figurename{Figura}
\renewcommand\tablename{Tabla} 
\renewcommand\chaptername{Cap�tulo}
\renewcommand\appendixname{Ap�ndice}
\renewcommand\abstractname{Resumen}



%%%%%%%%%%%%%%%%%%%%%%%%%%%%%%%%%%%%%%%%%%%%%%
%%% PORTADA
%%%%%%%%%%%%%%%%%%%%%%%%%%%%%%%%%%%%%%%%%%%%%%

\thispagestyle{empty}
\vspace*{1cm}
\hrule
\mbox{ }

\begin{center}
\begin{LARGE}
Trabajo Fin de M�ster: T�tulo del Trabajo \\ [0.3em]
\end{LARGE}
\end{center}

\hrule
\vspace*{1cm}
\begin{center}
 \includegraphics[width=0.3\textwidth]{imagenes/Logo_UNED.jpg} %[width=4cm,,keepaspectratio]
\end{center}

\vfill
\begin{center}
  {\Large {\bf Trabajo Fin de M�ster}}
\end{center}

\vfill
\begin{center}
  \textbf{Nombre del Alumn@} \\ [0.2em]
  Trabajo de investigaci�n para el \\ \vspace{10pt}
  M�ster en Lenguajes y Sistemas Inform�ticos\\ \vspace{10pt}
  Universidad Nacional de Educaci�n a Distancia\\ \vspace{10pt}
  \vspace{2cm}
  Dirigido por el \\ \vspace{10pt}
  \textbf{Prof. Dr. D. Nombre Direct@r} \\ [0.5em]
  Octubre 2018\\
\end{center}


%%%%%%%%%%%%%%%%%%%%%%%%%%%%%%%%%%%%%%%%%%%%%%
%%% AGRADECIMIENTOS
%%%%%%%%%%%%%%%%%%%%%%%%%%%%%%%%%%%%%%%%%%%%%%
\chapter*{Agradecimientos}
\label{sec:Agradecimientos}

\noindent Agradecimientos si procede.


%%%%%%%%%%%%%%%%%%%%%%%%%%%%%%%%%%%%%%%%%%%%%%
%%% RESUMEN
%%%%%%%%%%%%%%%%%%%%%%%%%%%%%%%%%%%%%%%%%%%%%%
\chapter*{Resumen}
\label{sec:Resumen}
\noindent En el presente trabajo se propone una arquitectura para la detecci�n del machismo en la red de microblogging Twitter. El objetivo es comprender los mecanismos y se�ales textuales que conlleven actitudes o lenguaje machista en castellano. Para ello, se compone un corpus que contiene texto con contenido y expresiones machistas utilizando como fuente de datos la red social Twitter. Para la creaci�n del corpus, se desarrollar� una herramienta que recopilar� los mensajes creados en Twitter que contengan distintos t�rminos que pueden conllevar comportamientos machistas. Las expresiones o t�rminos que se buscan son aquellas que, de un modo u otro, minusvaloran el papel de las mujeres en nuestra sociedad, incentiven el abuso o acoso hacia las mujeres, o no les permita expresarse libremente. Existen gran cantidad de expresiones y t�rminos que se utilizan a diario de modo consciente o no que minimizan el papel de la mujer en la sociedad. Expresiones como ``feminazi'', ``ni�ata'' o ``a fregar'' han sido utilizadas para recopilar los tweets. Asimismo, utilizando el corpus generado, se desarrolla un sistema de clasificaci�n que permite detectar lenguaje machista de un modo automatizado. Se emplean distintos algoritmo de clasificaci�n y se establecen dos l�neas base que sirven como punto de partida para la evaluaci�n del sistema.

Finalmente, se ha realizado una evaluaci�n exhaustiva del sistema de clasificaci�n sobre el corpus creado para determinar su rendimiento. Los resultados obtenidos muestran que el procedimiento empleado es capaz de detectar el machismo en el corpus utilizado con un buen acierto. Sin embargo, se han detectado distintas limitaciones y problemas debido a la aproximaci�n empleada.



%%%%%%%%%%%%%%%%%%%%%%%%%%%%%%%%%%%%%%%%%%%%%%
%%% ABSTRACT
%%%%%%%%%%%%%%%%%%%%%%%%%%%%%%%%%%%%%%%%%%%%%%
\chapter*{Abstract}
\label{sec:Abstract}
\noindent The present work defines a new architecture to detect sexism at Twitter. The main goal is to understand the mechanism and textual signals which imply sexist attitudes or language of this kind in spanish. We create a new corpus which contains text with expressions and sexist attitudes using the social network Twitter as main datasource. To create the corpus, a specific tool is developed to collect all the messages created at Twitter containing certains terms related to sexist attitudes. We search for expressions which underestimate the role of women in our society, encourage the harassment towards them or limit their freedom of speech. There are many expressions used in a daily basis that underestimate the role of women in society. Terms such as ``feminazi'', ``ni�ata'' o ``a fregar'' have been used to collect tweets in our corpus. We also develop a classification system which allows us to identify sexist language in an automatic way. The system employs several machine learning algorithms and establish two baselines used as reference points for the system's evaluation.

Finally, an extensive evaluation is perfomed on the corpus to determine the perform of our approach in the classification task. The results obtained shows this procedure is able to identify sexism properly in the corpus used. However, some limitations and problems have been detected due to the approach used.



%%%%%%%%%%%%%%%%%%%%%%%%%%%%%%%%%%%%%%%%%%%%%%
%%% INDICES
%%%%%%%%%%%%%%%%%%%%%%%%%%%%%%%%%%%%%%%%%%%%%%
\tableofcontents
\listoffigures
\listoftables


%%%%%%%%%%%%%%%%%%%%%%%%%%%%%%%%%%%%%%%%%%%%%%
%%% ENTRADAS A LOS FICHEROS CON LOS CAPITULOS
%%%%%%%%%%%%%%%%%%%%%%%%%%%%%%%%%%%%%%%%%%%%%%
%%%%%%%%%%%%%%%%%%%%%%%%%%%%%%%%%%%%%%%%%%%%%%
%%% INTRODUCCION
%%%%%%%%%%%%%%%%%%%%%%%%%%%%%%%%%%%%%%%%%%%%%%

\chapter{Introducci�n}
\fancyhead[RE]{\textsc{CAP�TULO} \thechapter. Introducci�n}
\label{ch:Introduccion}
\pagenumbering{arabic}

\section{Motivaci�n}
\label{sec:Motivacion}

\noindent Motivaci�n del trabajo a realizar.


El primer p�rrafo de cada secci�n no se indenta, los siguientes s�.
Las referencias a un art�culo del estado del arte se ponen as�: El trabajo  bla bla.

La presente plantilla es meramente orientativa. Tanto los cap�tulos como las secciones de cada cap�tulo son simplemente una gu�a para ayudar al alumno en el proceso de escritura del trabajo final del m�ster, pero en ning�n caso supone que los trabajos tengan que tener forzosamente esta estructura. La estructura del trabajo debe ser acordada por el alumno y su director/res, as� como adecuarse a la tem�tica abordada.

\section{Propuesta y objetivos}
\label{sec:PropuestaYObjetivos}

\noindent Que se ha realizado en el trabajo de fin de master, cuales eran los objetivos y breve resumen de los resultados obtenidos. Texto de prueba 3.

\section{Estructura del documento}
\label{sec:EstructuraDelDocumento}

\noindent Breve descripci�n de los cap�tulos del trabajo.

\begin{description}

\item \textbf{Cap�tulo 1. Introducci�n.} Este cap�tulo introduce los principales motivos que han llevado a la realizaci�n de este trabajo, as� como la problem�tica y el estado actual de la disciplina. Por �ltimo, se presentan las diferentes contribuciones del trabajo realizado.
\item \textbf{Cap�tulo 2. Estado del arte.} Este cap�tulo describe en mayor detalle la disciplina que nos ocupa, presentando su origen y su historia hasta el presente. Se muestran las t�cnicas actuales m�s utilizadas para resolver las tareas m�s relevantes del tema abordado, as� como sus debilidades.
\item \textbf{Cap�tulo 3. Sistema/M�todo/Caso de Estudio propuesto.}  En este cap�tulo se describe en profundidad el sistema/m�todo o caso de estudio propuesto.
\item \textbf{Cap�tulo 4. Evaluaci�n.} Este cap�tulo describe la metodolog�a utilizada para evaluar la propuesta realizada, a la vez que presenta los resultados obtenidos al evaluar el m�todo propuesto en diferentes tareas y sobre colecciones de evaluaci�n de distintos dominios.
\item \textbf{Cap�tulo 5. Discusi�n.} Este cap�tulo analiza y discute en profundidad los resultados obtenidos en la evaluaci�n presentada en el cap�tulo anterior.
\item \textbf{Cap�tulo 6. Conclusiones y trabajo futuro.} Este cap�tulo recopila las diferentes conclusiones extra�das del trabajo realizado, y propone algunas l�neas de trabajo futuro.
\end{description} 
%%%%%%%%%%%%%%%%%%%%%%%%%%%%%%%%%%%%%%%%%%%%%%%%%%%
%%% ESTADO DEL ARTE
%%%%%%%%%%%%%%%%%%%%%%%%%%%%%%%%%%%%%%%%%%%%%%%%%%%

\chapter{Estado del arte}
\fancyhead[RE]{\textsc{CAP\'ITULO} \thechapter. Estado del arte}
\label{ch:EstadoArte}

\noindent El presente cap�tulo tiene como objetivo presentar al lector la detecci�n del lenguaje machista en redes sociales. Para ello, se realizar� una revisi�n de los trabajos m�s relevantes en la tarea de detecci�n de lenguaje abusivo y machista, en los que se analizar�n los or�genes de esta tarea, las soluciones t�cnicas y las aportaciones m�s relevantes.

\section{Detecci�n de lenguaje o discurso del odio (\textbf{\textit{hate speech detection}}) }
\label{sec:Ejemplo_seccion}
La detecci�n del lenguaje machista o sexista est� muy relacionada con la detecci�n del lenguaje o discurso del odio en redes sociales. Existen numerosos trabajos donde se intenta detectar distintos tipos de lenguaje del odio, entre ellos el sexismo \cite{WATANABE2018,WaseemHovy2016,Georgios2018,Badjatiya2017,Zimmerman2018,Park2017,Waseem2016}. El lenguaje del odio se refiere al uso de lenguaje agresivo, violento u ofensivo hacia un grupo espec�fico de personas que comparten una propiedad en com�n, sea esta propiedad su g�nero, su raza, sus creencias o su religi�n \cite{Davidson2017}. Atendiendo a esta definici�n, se puede considerar la detecci�n del machismo como un caso particular del discurso del odio. Por ello, es muy interesante realizar una evaluaci�n de los trabajos realizados en esta l�nea de investigaci�n.

La detecci�n del lenguaje del odio es una linea de investigaci�n muy actual, el primer estudio evaluado data del a�o 2012 \cite{Xiang2012}. En este articulo se emplea un modelo de detecci�n de temas o categor�as (\textit{topic modelling}) que explota la concurrencia de palabras para la creaci�n de atributos o \textit{features} que alimentar�n un algoritmo de clasificaci�n de aprendizaje de m�quina o \textit{machine learning}. En la mayor�a de trabajos previos se empleaban soluciones basadas en patrones para la clasificaci�n de tweets. De este modo, este art�culo supone un paso muy importante hacia la automatizaci�n y a los sistemas basados en algoritmos de \textit{machine learning}. Adem�s, durante la etapa anterior a este art�culo, el uso de expresiones coloquiales y soeces en redes sociales hace dif�cil establecer las fronteras entre el uso de lenguaje ofensivo que no tiene como objetivo despreciar a ning�n grupo de personas y el lenguaje del odio \cite{Davidson2017} utilizando patrones extra�dos de la utilizaci�n del lenguaje.

Durante los �ltimos tres a�os, se han sucedido los art�culos en la tem�tica y ha aumentado considerablemente la producci�n cient�fica en este campo. En \cite{WaseemHovy2016} se aporta el primer corpus de referencia anotados que se utilizar� posteriormente en \cite{Waseem2016,Georgios2018,Badjatiya2017,Zimmerman2018,Park2017}. Est� compuesto por 16.000 \textit{tweets} etiquetados para mensajes sexistas, racistas o sin contenido ofensivo. En este primer trabajo, se sientan las bases de las soluciones aplicadas en el resto de art�culos, se utilizan atributos como los \textit{unigramas, bigramas, trigramas} y \textit{cuatri-gramas} y un algoritmo de regresi�n log�stica para la clasificaci�n.

En el art�culo desarrollado por el mismo autor \cite{Waseem2016} se propone una soluci�n similar pero se ampl�a el corpus en 4033 \textit{tweets} y se utiliza una plataforma de \textit{crowdsourcing} para anotar los mensajes. Achacan el empeoramiento de los resultados al posible sesgo que se produce en \cite{WaseemHovy2016} ya que los \textit{tweets} solo fueron etiquetados por los autores �nicamente.

En el resto de art�culos que eval�an su propuesta utilizando el corpus desarrollado por \cite{Waseem2016}, se utilizan redes neuronales para la tarea de clasificaci�n y, en algunos, en la etapa de preprocesamiento. En la soluci�n propuesta por \cite{Zimmerman2018} se aplican redes neuronales convolucionales (\textit{CNN, Convolutional Neural Network}) para codificar el texto y extraer los atributos que se utilizar�n para el clasificador final, basado tambi�n en CNNs. Esta t�cnica permite tener en cuenta la posici�n de la palabra (su contexto) para extraer los atributos de cada \textit{tweet}. Esta misma idea junto con el uso de redes neuronales recurrentes (\textit{RNN, Recurrent Neural Network}) se utiliza en \cite{Badjatiya2017} para obtener los atributos en la etapa de procesamiento. En ambos art�culos se consiguen mejorar los resultados alcanzados por \cite{Waseem2016}.

En \cite{Georgios2018} se propone un modelo basado en RNNs para abordar el problema. Adem�s se explora la idea de utilizar atributos como la tendencia al racismo o sexismo utilizando el historial de los usuarios. Se demuestra como el uso de este tipo de atributos mejora notablemente los resultados. Esta misma idea se utiliza en \cite{Chatzakouy2017} donde se detectan cuentas agresivas estudiando al usuario y su red de seguidores.

En todos los art�culos revisados anteriormente, se trata el problema como una clasificaci�n m�ltiple donde el texto se puede clasificar seg�n las etiquetas racismo, sexismo o ninguno. Sin embargo, se podr�a resolver el problema con un doble clasificador, el primero clasifica si el texto contiene lenguaje abusivo o no y el segundo realizar�a la tarea de clasificar en contenido sexista o racista \cite{Park2017}.

Un desaf�o importante en la detecci�n del lenguaje del odio en redes sociales es la separaci�n entre lenguaje ofensivo y el lenguaje que incita o promueve el odio. Davidson \cite{Davidson2017} aporta un corpus etiquetado de 25.000 \textit{tweets} para diferenciar entre estos 2 tipos de lenguaje. En su trabajo, se propone un modelo similar a \cite{Waseem2016} donde se ponen de manifiesto las dificultades de esta soluci�n para tener en cuenta el contexto de las palabras. De este modo, si se utilizan palabras que pueden expresar odio (por ejemplo, "\textit{gay}") en un contexto positivo, hay muchas probabilidades de que el sistema detecte odio en el texto. Los resultados ser�n mejorados posteriormente en \cite{WATANABE2018} donde se ampliar� el n�mero de\textit{features} y se utilizar� un algoritmo basado en �rboles de decisi�n para la tarea de clasificaci�n.











\section{Ejemplo secci�n}
\label{sec:Ejemplo_seccion}

\subsection{Ejemplo subsecci�n}
\label{sec:Ejemplo_subSeccion} 
%%%%%%%%%%%%%%%%%%%%%%%%%%%%%%%%%%%%%%%%%%%%%%%%%%%
%%% Herramientas utilizadas
%%%%%%%%%%%%%%%%%%%%%%%%%%%%%%%%%%%%%%%%%%%%%%%%%%%

\chapter{Herramientas utilizadas}
\fancyhead[RE]{\textsc{CAP�TULO} \thechapter. Herramientas utilizadas}
\label{ch:Herramientas}

\noindent En este cap�tulo se describen en profundidad las distintas herramientas evaluadas para la creaci�n del sistema propuesto. Adem�s, se exponen los motivos por los que se han elegido frente a otras alternativas disponibles.

\section{Crawler}
\label{sec:Ejemplo_seccion}
\subsection{Amazon Web Services}{}
\label{sec:Ejemplo_seccion}
\noindent AWS es una creciente unidad dentro la compa��a Amazon.com que ofrece una importante variedad de soluciones de Cloud Computing tanto PYMES como a grandes empresas a trav�s de su infraestructura interna, siendo la marca m�s utilizada actualmente en el mercado de la nube con casi un 40\% de cuota de mercado \cite{AWScuota}. Amazon ofrece unos servicios en la nube p�blica mediante una tarificaci�n de precios en funci�n del tiempo de uso, anchos de banda consumidos, etc. Por lo tanto, su gran ventaja competitiva es ofrecer unos recursos de infraestructura y plataforma poco asumibles a la mayor�a de empresas para el periodo que se requiera. 

Los clientes de AWS tan s�lo deben pagar lo que usen del servicio, de esta manera, obtener unos potentes servidores con una plataforma determinada, un espacio de almacenamiento o una gran base de datos supone la adquisici�n de un hardware que no se aproveche todo el tiempo, que tan s�lo interese para un periodo determinado y satisfacer una necesidad puntual, prescindiendo de importantes inversiones en infraestructura. Orientado a empresas, se adapta con total flexibilidad y escalabilidad a las necesidades de cloud que tenga el cliente, mediante un acuerdo de nivel de servicio, se especifica el nivel de compromiso del servicio, disponibilidad y ofrece un punto de confianza que otros proveedores de nube p�blica no proporcionan, dato que le da ventaja frente a sus competidores.

Dado que ha sido pionero en el sector y posee una gran cantidad de desarrolladores que trabajan para mejorar el servicio, desde su publicaci�n en 2006, ha sido l�der en el sector por delante de Google App Engine, Azure de Microsoft, Alibaba, etc \cite{AWScuota}. Siempre ha ido un paso por delante y le ha permitido innovar en el sector y ofrecer unos precios muy competitivos, soluciones para todos los gustos e importantes acuerdos con Microsoft, IBM y HP como estrategias de marketing para ofrecer software y plataformas propietarias (adem�s de software libre que fue lo primero que se ofrec�a con plataformas Linux) en sus im�genes de m�quinas virtuales. En la siguiente figura se puede ver un resumen de los servicios de AWS:

\begin{center}
	\includegraphics[width=0.9\textwidth]{imagenes/aws_services.jpg} %[width=4cm,,keepaspectratio]
	\captionof{figure}{Resumen de servicios AWS}	
\end{center}


Los diferentes servicios de AWS se incrementan con el paso del tiempo, siendo EC2, S3 y Lambda los que m�s peso tienen en el presente proyecto:

\begin{itemize}
	\itemsep0em 
	\item Amazon Elastic Compute Cloud (EC2): proporciona servidores virtuales escalables. Proporciona las capacidades de Cloud Computing a sus clientes de manera que permite una configuraci�n y administraci�n de las capacidades de m�quinas virtuales que se solicitan a la nube, pudiendo pagar tan s�lo el tiempo de computaci�n. Actualmente existen numerosos tipos de instancias con caracter�sticas hardware distintas seg�n los requisitos del usuario.
	\item Amazon Simple Storage Service (S3): proporciona un Web Service basado en el almacenamiento online para aplicaciones. Este almacenamiento en Internet proporciona una simple interfaz web, como su nombre indica, que puede ser usada para almacenar grandes cantidades de datos en cualquier momento desde cualquier sitio, dando acceso confiable y seguro con SLA, altamente escalable, r�pido y barato en la infraestructura de Amazon. F�sicamente, los datos est�n distribuidos por los Data Center de Amazon, pero es algo que permanece ajeno al cliente y de lo que no debe preocuparse (escalabilidad). Su integraci�n con EC2 es esencial para que las im�genes de m�quinas virtuales puedan trabajar con datos y objetos almacenados en S3 y tener un espacio donde los desarrolladores puedan trabajar c�modamente incluso poder solicitar m�s espacio temporal para las m�quinas o disponer de varios ``buckets'' donde compartir datos entre instancias.
	\item AWS Lambda: se trata de un servicio de computaci�n sin servidor. Este servicio permite ejecutar c�digo sin aprovisionar ni administrar servidores, pagando �nicamente por el tiempo de c�mputo que se consuma. De este modo, este servicio permite que se AWS quien se encargue de la administraci�n de las m�quinas y el usuario �nicamente trabaje en el c�digo que se ejecuta.
\end{itemize}

La segunda plataforma de cloud computing m�s importante a nivel mundial es Azure, propiedad de la empresa Microsoft. En este caso, no se ha elegido Microsoft Azure porque ya se contaba con un conocimiento previo en el uso de los servicios de AWS. Adem�s, AWS cuenta con servicios de computaci�n serverless, como AWS Lambda, muy �tiles para la realizaci�n del crawler. 

\subsection{Twitter API y rtweet}{}
\label{sec:Ejemplo_seccion}
\noindent Twitter proporciona m�ltiples APIs para facilitar el acceso a los datos de su plataforma. De todas ellas, la necesaria para crear el corpus objetivo ser�a el API REST de Twitter \cite{TwitterAPI}. En concreto, es necesario utilizar la funcionalidad Tweet Search que permite realizar b�squedas de los tweets generados en la plataforma seg�n distintos par�metros de b�squeda.

Dentro del API existen 3 tipos de cuenta seg�n la cantidad de informaci�n disponible para consulta: Standard Search, Premium Search y Enterprise Search. De todas ellas, solamente la primera es gratuita por lo que ser� la utilizada durante el proceso de generaci�n del corpus. Es importante se�alar que este tipo de b�squeda presenta algunas limitaciones. Las dos m�s importantes ser�an la existencia de una ventana temporal de consulta limitada a 7 d�as anteriores y, por otra parte, la limitaci�n de descarga de tweets a 18.000 cada 15 minutos.

Para recopilar la informaci�n de Twitter, se ha utilizado la herramienta rtweet \cite{rtweet-package}. Se trata de un cliente del lenguaje de programaci�n R para acceder al API de Twitter. Este paquete facilita mucho las tareas habituales como la b�squeda de tweets.

Existen varias alternativas a rtweet como tweetpy \cite{Tweetpy} para el lenguaje de programaci�n Python o twitteR. Se ha optado por rtweet porque ambas alternativas est�n m�s desactualizadas y son proyectos mucho m�s inactivos.


\section{Preprocesado y tokenizaci�n}
\label{sec:Ejemplo_seccion}

\noindent En la etapa inicial para la clasificaci�n de textos, se aplican distintas t�cnicas que permitan Extraer los atributos o \textit{features} necesarias para realizar una representaci�n fiel del texto y que permitan la utilizaci�n de un algoritmo de clasificaci�n. Existen multitud de procedimientos aplicables en esta etapa como la tokenizaci�n, el reconocimiento de entidades nombradas, el etiquetado sint�ctico y morfol�gico:

\begin{itemize}
	\itemsep0em 
	\item Tokenizaci�n: Permite separar cada palabra o s�mbolo del corpus en unidades independientes (como palabras) que pueden ser almacenadas para su posterior procesado.
	\item Reconocimiento de entidades nombradas: Tarea que permite clasificar fragmentos de texto en categor�as predefinidas, como personas, organizaciones, lugares, expresiones de tiempo y cantidades.
	\item Etiquetado sint�ctico: Proceso en el que se busca sobre el espacio de todas las posibles combinaciones de las reglas gramaticales definidas para encontrar la estructura de una oraci�n.
	\item Etiquetado morfol�gico: En este proceso se le asigna a cada palabra su funci�n dentro del corpus utilizado. Normalmente, se utilizan 8 etiquetas distintas en la mayor�a de los idiomas utilizados en Europa: nombre, verbo, pronombre, preposici�n, adverbio, conjunci�n, part�cula y articulo.
\end{itemize}

Para aplicar este tipo de t�cnicas, existen gran cantidad de proyectos o librer�as de computaci�n disponibles. Algunas de las m�s utilizadas son las siguientes:

\begin{itemize}
	\itemsep0em 
	\item Freeling \cite{carreras04}: Es una librer�a que soporta el lenguaje espa�ol y se utiliza en \cite{Frenda2018}. Pese a que tiene mucha de las caracter�sticas que se necesitan, tiene una menor comunidad y est� menos extendido que algunas del resto de las herramientas.
	\item Stanford Parser \cite{StanfordParser}: Se trata de una librer�a desarrollada por el grupo de trabajo de NLP de la universidad de Stanford.
	\item TweetNLP \cite{TweetNLP}: Librer�a desarrollada espec�ficamente para le procesado de tweets. Su uso no est� muy extendido.
	\item Spacy \cite{Spacy}: Se utiliza en \cite{Waseem2016} y permite aplicar las t�cnicas de procesado de un modo eficiente.
	\item NLTK \cite{NLTKweb}: Se trata de la librer�a m�s extendida para el preprocesamiento, se utiliza en \cite{Zimmerman2018,Davidson2017,Frenda2018}.
\end{itemize}

De todas las herramientas listadas, se ha optado por la librer�a NLTK. Se trata de una librer�a muy extendida que cuenta con una gran comunidad y permite un desarrollo muy �gil. Algunas librer�as como Freeling o Stanford Parser requieren varias dependencias para poder ser utilizadas.

La mejor alternativa a NLTK considerada ser�a Spacy. Su uso est� aumentando y su funcionamiento es muy similar ya que ambas est�n desarrolladas en Python. Se ha optado por NLTK porque a d�a de hoy sigue siendo m�s utilizada.

\subsection{NLTK: Natural Language Toolkit}{}
\label{sec:Ejemplo_seccion}
\noindent NLTK \cite{NLTKweb} es una librer�a que define una infraestructura en la que crear programas para el procesado del lenguaje natural (NLP, ``\textit{Natural language processing}'') en ``\textit{Python}''. Provee la estructura b�sica para representar datos relevantes para el procesado del lenguaje natural, interfaces para realizar tareas como el etiquetado del discurso (POS, ``part-of-speech tagging''), etiquetado sint�ctico y clasificaci�n de texto.

Esta librer�a fue desarrollada originalmente en el a�o 2001 como parte de un curso de ling��stica computacional en la Universidad de Pennsylvania. Desde entonces, ha sido desarrollada y mejorada por distintos contribuidores al tratarse de un proyecto libre. Actualmente, NLTK es utilizado en gran cantidad de investigaciones y supone un est�ndar muy importante para realizar tareas relacionadas con NLP. Est� compuesto por una cantidad importante de m�dulos que pueden ser invocados desde un programa escrito en Python. En la siguiente figura se recogen los m�s importantes \cite{Bird2009}:

\begin{center}
	\includegraphics[width=0.9\textwidth]{imagenes/NLTK_modules-1.jpg} %[width=4cm,,keepaspectratio]
	\captionof{figure}{M�dulos NLTK}	
\end{center}

\section{Scikit-learn}
\label{sec:Ejemplo_seccion}


\noindent Scikit-learn \cite{Pedregosa2011}  es un proyecto que provee una librer�a de aprendizaje de m�quina para el entorno de programaci�n ``Python''. El objetivo principal de esta librer�a es establecer un conjunto de herramientas dentro de un entorno de programaci�n que sea accesible a usuarios no expertos. Esta librer�a incluye algoritmos cl�sicos de aprendizaje de m�quina, herramientas para la selecci�n, evaluaci�n de modelos y preprocesado. Todos los objetos dentro de la librer�a comparten una API b�sica compuesta por 3 interfaces complementarias: ``estimators'' que permiten construir y ajustar modelos, ``predictors'', para realizar predicciones, y ``transformers'', que permiten realizar conversiones a los datos.

La mayor parte de modelos de aprendizaje supervisados o funciones auxiliares relacionadas con el procesado de datos utilizados en el presente trabajo est�n implementados o han sido desarrollados con ayuda de funciones disponibles en la librer�a scikit-learn.


\subsection{``Estimators''}
\label{sec:Ejemplo_subSeccion}

\noindent La interfaz ``estimator'' define objetos y provee de un m�todo ``fit'' para ajustar un modelo a los datos de entrenamiento. Todos los algoritmos supervisados y no supervisados implementados en la librer�a son tratados como objetos implementando esta interfaz. Otro tipo de tareas relacionadas con el aprendizaje de m�quina como la selecci�n de atributos o m�todos para la reducci�n de la dimensionalidad tambi�n utilizan el interfaz ``estimator''.

La inicializaci�n de un ``estimator'' y el ajuste de un modelo a los datos de entrenamiento est�n diferenciados en la librer�a. Un ``estimator'' se puede inicializar un con conjunto de par�metros de entrada (por ejemplo, el par�metro C para SVM) y, posteriormente, se utiliza el m�todo ``fit'' para realizar el proceso de ajuste a los datos de entrenamiento. En el siguiente c�digo se ilustra esta funcionalidad:

\begin{lstlisting}[language=Python]
from sklearn.ensemble import RandomForestClassifier
rf = RandomForestClassifier(n_estimators=250)
rf.fit(X_train, y_train)
\end{lstlisting}

En el c�digo anterior, primero se inicializa un ``estimator'' estableciendo el argumento ``n\_estimators''. Tras esto, se realiza una llamada al m�todo ``fit'' para realizar el ajuste utilizando los datos de entrenamiento.


\subsection{``Predictors''}
\label{sec:Ejemplo_subSeccion}

\noindent La interfaz ``predictor'' extiende la funcionalidad del ``estimator'' a�adiendo el m�todo ``predict''. Este m�todo devuelve un vector de predicciones tomando como entrada una matriz con los datos de testeo. Ampliando el ejemplo anterior:

\begin{lstlisting}[language=Python]
y_pred = rf.predict(X_test)
\end{lstlisting}


\subsection{``Transformers''}
\label{sec:Ejemplo_subSeccion}

\noindent Antes de aplicar un m�todo de clasificaci�n supervisada, suele ser habitual realizar filtrados o modificaciones en los datos, para ello, ``scikit-learn'' implementa la interfaz ``transformer''.

Esta interfaz define el m�todo ``transform'' que toma como entrada una matriz de datos y devuelve como salida una versi�n transformada de estos datos. Algunas de las transformaciones m�s comunes pueden ser la selecci�n de atributos, preprocesado o m�todos de reducci�n de dimensionalidad. Un ejemplo de preprocesado podr�a ser el estandarizado de un conjunto de datos:

\begin{lstlisting}[language=Python]
from sklearn.preprocessing import StandardScaler
scaler = StandardScaler()
scaler.fit(X_train)
X_train = scaler.transform(X_train)
\end{lstlisting}


\subsection{``Pipelines y selecci�n de modelos''}
\label{sec:Ejemplo_subSeccion}

\noindent ``Scikit-learn'' permite componer nuevos ``estimators'' utilizando otros, lo que permite crear flujos de trabajo completos en un �nico objeto. Este tipo de tarea se puede realizar de dos modos: mediante ``Pipeline'' utilizando un modelo secuencial o mediante ``FeatureUnion''.

Los objetos ``Pipeline'' encadenan ``estimators'' en un �nico objeto. Esto permite crear flujos de trabajo siguiendo un n�mero fijo de pasos, por ejemplo: extracci�n de atributos, reducci�n de dimensionalidad, ajuste de un modelo y realizaci�n de de predicciones.

Los objetos ``FeatureUnion'' combinan multiples ``transformers'' en uno �nico y concatena los resultados. De este modo, este tipo de objeto es capaz de realizar transformaciones distintas sobre el mismo conjunto de datos o sobre una parte del mismo.

Ambos objetos pueden ser combinados para crear flujos de trabajo m�s complejos. Por ejemplo, en el siguiente c�digo se combinan dos ``Pipeline'' utilizando ``FeatureUnion'' y se a�ade un �ltimo paso ``clf'' que a�ade un clasificador.
\begin{lstlisting}[language=Python]
Pipeline([('feature-union', 
	FeatureUnion([('text-features', 
	text_pipeline), 
	('other-features', preprocess_pipeline)])),
	('clf', LogisticRegression(penalty = ``L2'')
	])

\end{lstlisting}


En ``scikit-learn'' es posible realizar selecci�n de modelos mediante el meta-estimador ``GridSearchCV''. Este m�todo toma como entrada un ``estimator'' cuyos par�metros de entrada deben de ser optimizados. Para ello, se definen todos los valores que se deben de tener en cuenta en el proceso para cada par�metro de entrada.


%%%%%%%%%%%%%%%%%%%%%%%%%%%%%%%%%%%%%%%%%%%%%%%%%%%
%%% CORPUS
%%%%%%%%%%%%%%%%%%%%%%%%%%%%%%%%%%%%%%%%%%%%%%%%%%%

\chapter{Corpus (MachismoTwitter, DAMT (detecci�n autom�tica machismo twitter, Twittmach, Machistwitter, Machistlab)}
\fancyhead[RE]{\textsc{CAP�TULO} \thechapter. Corpus}
\label{ch:Corpus}

%%%%%%%%%%%%%%%%%%%%%%%%%%%%%%%%%%%%%%%%%%%%%%%%%%%%%%%%%%%%%%%%%%%%%%%%%%%%%%%%
CORPUS (MachismoTwitter, DAMT (detecci�n autom�tica machismo twitter, Twittmach, Machistwitter, Machistlab):
1-Motivaci�n y definici�n del machismo (las definiciones de la gu�a de anotaci�n)
1.1 - Como he seleccionado esas expresiones machistas (meter referencias). Proceso de selecci�n de las etiquetas machistas.


2-Crawler (meter como se hace el crawler, Evernote->TFM->Notas):
Descartadas las expresiones siguientes por tener menos de 100 tweets.
"acabaras sola"	50
"hombre que te aguante"	37
"obsesionada con el machismo"	8
"pareces una fulana"	5
"no ha probado un hombre"	1



5- Resultados:
5.0- Campos disponibles en el API de Twitter (En el trabajo de MW)
5.1-An�lisis preliminar del corpus: N�mero de tweets


3 - Etiquetado del corpus (las definiciones de la gu�a de anotaci�n)
3.1-Ejemplos de la gu�a de anotaci�n
3.2-Problemas encontrados: ambiguedad (tweets en los que tuvimos desacuerdo total), citaciones de expresiones machistas, redefinici�n de las reglas de etiquetado por fallo del kappa en el 20%
3.3 - Herramienta utilizada para el etiquetado (google drive)
3-4 - Cruce entre el resultado de los anotadores y el resto de campos
3.4 - Kappa alcanzado, acuerdo entre anotadores (meter m�tricas de Excel).
5.2-An�lisis de la exploraci�n del corpus de spyder

6 - Adjuntar como anexo la gu�a de anotaci�n
%%%%%%%%%%%%%%%%%%%%%%%%%%%%%%%%%%%%%%%%%%%%%%%%%%%%%%%%%%%%%%%%%%%%%%%%%%%%%%%%%%%%%%


\noindent Este cap�tulo describe la metodolog�a utilizada para evaluar la b�squeda y creaci�n de un corpus en castellano con el objetivo de detectar lenguaje machista en la red de microblogging Twitter. De este modo, se realiza un corpus utilizando un conjunto de t�rminos en castellano que, por lo general, pueden llevar a actitudes machistas. Esto permitir� el desarrollo de un sistema para detectar este tipo de comportamientos en Twitter.

Idealmente, el conjunto obtenido en el trabajo permitir� realizar un sistema de clasificaci�n autom�tico para la detecci�n de tweets con contenido machista. Por tanto, el objetivo principal es la creaci�n un corpus que contenga texto con contenido machista. Para ello se realiza una b�squeda en la red social Twitter con distintos t�rminos que pueden llevar a comportamientos machistas.

\section{Machismo en Twitter}
\label{sec:Met_Eval}
\noindent En este apartado, se van a definir todas las caracter�sticas que deben de presentar los textos seleccionados para considerarse machistas y la lista de t�rminos o expresiones seleccionados para generar el corpus. Como primer paso, se han estudiado diversas referencias para recopilar aquellas expresiones m�s comunes que pueden conllevar a comportamientos machistas o sexistas. Posteriormente, se han identificado las expresiones m�s representativas y se han realizado diversas b�squedas para evaluar la cantidad de contenido machista que contienen dichas expresiones en Twitter. Finalmente, se han obtenido de la red social Twitter los mensajes que conten�an estas expresiones utilizando su API. Durante este estudio solo se han considerado expresiones en castellano.

Las expresiones o t�rminos que se buscan son aquellas que, de un modo u otro, minusvaloran el papel de las mujeres en nuestra sociedad, incentiven el abuso o acoso hacia las mujeres o no les permita expresarse libremente.
Existen gran cantidad de expresiones y t�rminos que se utilizan a diario de modo consciente o no que minimizan el papel de la mujer en la sociedad. En \cite{TwitterSexism} la periodista Ana Isabel Bernal-Trivi�o propone a trav�s de un hilo en su cuenta de Twitter recopilar frases de violencia machista que las mujeres hayan sufrido en alg�n momento. Estas expresiones tienen mucho valor para el an�lisis pues un gran n�mero de mujeres aporta su experiencia personal frente al machismo. En muchos tweets se repiten t�rminos como ?ni�ata? o ?a fregar? que se utilizan para referirse a las mujeres de forma despectiva.

Existen refranes, dichos populares y t�picos que se utilizan diariamente refuerzan la idea de que las mujeres valen menos \cite{FrasesMachistas}. Por ejemplo, la expresi�n ?Mujer al volante, peligro constante? �Mujer ten�a que ser!? minusvalora la habilidad de las mujeres para realizar una tarea espec�fica solo por ser mujeres. Otras muchas expresiones machistas se recopilaron en un trabajo realizado en un instituto de Albacete \cite{ElPais}. Expresiones como ?marimacho? o ?�Inform�tica? �No prefieres bailar?? muestran el machismo que sufren muchas mujeres desde temprana edad.

Pese a que, como se ha presentado, existen gran cantidad de expresiones que conllevan actitudes machistas generalmente, ha sido necesario recoger ciertos t�rminos m�s generales que permitan establecer una buena relaci�n entre tweets con contenido machista y aquellos que no presentan este tipo de lenguaje. En este caso, se han seleccionado las siguientes expresiones: ``feminazi'', '``a la cocina'', '``a fregar'', ``marimacho'', ``ninata'', '``mujer tenias que ser'', '``las feministas'',``en tus dias'', ``zorra'','``como una mujer'', '``como una nina'', '``pareces una fulana'', '``pareces una puta'','``no ha probado un hombre'','``loca del'', '``obsesionada con el machismo'', '``para ser mujer'','``para ser chica'','``hombre que te aguante'','``acabaras sola'', ``mojigata'', '``mucho feminismo pero'','``mujer al volante'', '``las mujeres no deberian'', '``A las mujeres hay que'', '``odio a las mujeres'','``las mujeres de hoy en dia'', ``nenaza'', ``lagartona''.

% https://www.tablesgenerator.com/
% Please add the following required packages to your document preamble:
% \usepackage{graphicx}
\begin{table}[]
	\centering
	\resizebox{\textwidth}{!}{%
		\begin{tabular}{lll}
			N�mero de t�rmino & Texto                         &  \\
			1                 & ``feminazi"                    &  \\
			2                 & ``a la cocina"                 &  \\
			3                 & ``a fregar"                    &  \\
			4                 & ``marimacho"                   &  \\
			5                 & ``ninata"                      &  \\
			6                 & ``mujer tenias que ser"        &  \\
			7                 & ``las feministas"              &  \\
			8                 & ``en tus dias"                 &  \\
			9                 & ``zorra"                       &  \\
			10                & ``como una mujer"              &  \\
			11                & ``como una nina"               &  \\
			12                & ``pareces una fulana"          &  \\
			13                & ``pareces una puta"            &  \\
			14                & ``no ha probado un hombre"     &  \\
			15                & ``loca del"                    &  \\
			16                & ``obsesionada con el machismo" &  \\
			17                & ``para ser mujer"              &  \\
			18                & ``para ser chica"              &  \\
			19                & ``hombre que te aguante"       &  \\
			20                & ``acabaras sola"               &  \\
			21                & ``mojigata"                    &  \\
			22                & ``mucho feminismo pero"        &  \\
			23                & ``mujer al volante"            &  \\
			24                & ``las mujeres no deberian"     &  \\
			25                & ``A las mujeres hay que"       &  \\
			26                & ``odio a las mujeres"          &  \\
			27                & ``las mujeres de hoy en dia"   &  \\
			28                & ``nenaza"                      &  \\
			29                & ``lagartona"                   & 
		\end{tabular}%
	}
	\caption{T�rminos elegidos para corpus machista}
	\label{my-label}
\end{table}

El t�rmino ?feminazi? es una forma de relacionar, de forma claramente despectiva, al feminismo con el nazismo. De este modo, es una palabra utilizada ampliamente en redes sociales y foros de toda Espa�a que conllevan actitudes machistas. En Twitter se encuentran mensajes como:

	``\textit{Uy habl� de inventos la feminazi}''
	
	
	``\textit{Pero que violento!!! De cinco billetes, solo en uno aparece una mujer, esto es inaceptable!!! \#feminazi}''

Sin embargo, existen otros ejemplos que se utiliza la expresi�n sin conllevar actitudes machistas:

``\textit{Si quieres pensar que el origen del hashtag deviene de la b�squeda del rigor hist�rico, a tope. Est�s en tu derecho, igual que yo en un texto de opini�n de valorarlo totalmente al contrario. �No te convencen mis argumentos? Intentemos debatir. �Usas "feminazi"? Te quedas solo}''

Los t�rminos ``A la cocina'' y ``A fregar'' se utilizan de modo despectivo para denigrar a la mujer o restar importancia a sus argumentos. Como en el caso del t�rmino anterior, se ha comprobado que existe una relaci�n aceptable entre mensajes machistas y aquellos que no conllevan estas actitudes. Algunos ejemplos de contenido machista:


``\textit{Hay que anunciar por megafonia que te marches a tu casa a fregar  que estas  haciendo el rid�culo .tu y las borregos}''


``\textit{De los relatos corazon... anda a la cocina a hacee algo productivo antes de seguir diciendo pavadas}''

Ejemplos que no expresan actitudes machistas:


``\textit{Me dio sed, pero me da miedo ir a la cocina}''


``\textit{Me molesta que a esta hora me d� hambre porque me da miedo ir a la cocina por algo para comer}''


Las palabras ``Marimach'' y ``Ni�ata'' se utilizan como calificativos despectivos a la mujer. El primer t�rmino se emplea para indicar que una mujer tiene rasgos propios de hombre mientras que el segundo se emplea para denotar inmadurez. Ejemplos despectivos ser�an los tweets encontrados:


``\textit{Es un travesti o una marimacho pasada de anabolicos?}''


``\textit{Quien es esta puta ni�ata para exigir nada?}''


Las expresiones ``Mujer ten�as que ser'' y ``Las Feministas'' se utilizan para descalificar a las mujeres por el hecho de pertenecer a un colectivo. Ejemplos encontrados en Twitter son los siguientes:


``\textit{Mujer tenias que ser para dar una respuesta tan pobre}''


``\textit{Ya salieron las feministas a dar la cara x @ aunque se lo fuma ella sola a el gordo sucio  a su segunda @ la anorexica}''


La expresi�n ``en tus d�as'' se utiliza como descalificativo para referirse a las mujeres por tener el periodo:



``\textit{Est�s insoportable en tus d�as.}''


``\textit{Seguro era porque andabas en tus d�as. xd}''


El t�rmino ?zorra? se utiliza continuamente en Twitter con actitudes machistas:



``\textit{zorra se puede saber pork no me sigues?}''


``\textit{No le presten atencion. como yo hago  a cualquier otro zorra}''

Otra expresi�n que minusvalora a las mujeres es ``como una mujer'' para comparar una acci�n que se realiza peor por ser mujer:


``\textit{Un hombre sin barba es como una mujer sin nalgas}''


``\textit{\#UnMadrazoPara Cesar Gaviria, que habla como una mujer en proceso de parto.}''


Otra expresi�n similar a la anterior ser�a ``como una ni�a'':


``\textit{No te vas a ir de aqu� sin decirme qu� sitio es ese. Y m�s te vale responder y no comportarte como una ni�a peque�a, ni�a peque�a.}''


El t�rmino ``nenaza'' se utiliza tambi�n como descalificativo en Twitter:

``\textit{A mi me ha pasado igual. Los lobos vestidos de corderos que lo mismo quieren matar fachas que lloran como una nenaza no puedo con ellos}''

``\textit{@marianorajoy ha perdido la Moncloa como una nenaza  porque no ha sabido defender como hombre y estadista a la Naci�n. Se va como los cobardes. Por la puerta de atr�s. El peor presidente de Espa�a }''

La expresi�n ``pareces una fulana'' se utiliza como descalificativo por la apariencia f�sica de las mujeres:


``\textit{Te acabo de ver en el Telenoticiasde TV3. Maquillate mejor, pareces una fulana.}''

``\textit{�L�vate esa cara que pareces una fulana!}''


Otra expresi�n muy similar a la anterior ser�a ``pareces una puta'':


``\textit{Deja de gritar hermano, pareces una puta}''

``\textit{Por fin no pareces una puta paleta que habla sin may�sculas}''


Otra expresi�n machista utilizada en Twitter es ``no ha probado un hombre'':


``\textit{que se pueda esperar de una mamerta  @ktikariza que no lee no estudia va a la provincia  tomar cerveza una millenian que no ha probado un hombre, el hambre o la guerra, estas juventud esta empecinada a decir y tener too regalado a costa de si }''

``\textit{pobre mujer, no ha probado un hombre en su vida.}''

La expresi�n ``loca del'' se utiliza de forma despectiva hacia las mujeres:

``\textit{Rosa D�ez o mejor llamada la loca del co�o }''

``\textit{Odio a esta clase de feministas, loca del co�o.}''

La frase ``obsesionada con el machismo'' se utiliza para restar importancia a la denuncia del machismo que realizan las mujeres:


``\textit{T�pica obsesionada con el machismo, el patriarcado y franco. De ah� no la sacas. No evoluciona}''

``\textit{Feliz d�a de la gente tontaca obsesionada con el machismo}''

Otra expresi�n que minusvalora las habilidades de la mujer ser�a ``para ser mujer'':


``\textit{Tomar�s hormonas para ser mujer?}''

``\textit{La Pamela es como una comunista, siempre poniendo problemas.
	Y malas relaciones.
	Se cree due�a de la verdad.
	Es una rota ordinaria, inculta .
	Le falta mucho para ser mujer decente.
	Qu� la corten de la TV luego y contraten gente culta.
	Adem�s, est� guatona y fea. 
}''


Otra expresi�n similar a la anterior ser�a ``para ser chica'':

``\textit{Daniella Ch�vez solo quiere pantalla, si vas a v�rtigo sabes que eres parte de la rutina de yerko, ademas Daniella demostr� en el cap�tulo que para lo �nico que sirve es para ser chica Playboy, y no soy machista, solo soy un simple espectador \#YerkoPuchento }''

``\textit{Sos un poco masculina como para ser chica, �sab�as?}''

Otra expresi�n machista utilizada en Twitter es ``hombre que te aguante'':


``\textit{Ching�n el hombre que te aguante hasta en tus d�as. }''

``\textit{Damaris, c�sate con tu "marido (Ana)" porque no vas a encontrar un hombre que te aguante.}''


Otra expresi�n machista utilizada en Twitter es ``acabaras sola'':


``\textit{\#SomosLaAudiencia21F  SOFIA:   � eres pat�tica, no, lo siguiente �  Deja a ALEJANDRO en paz de una vez. 
	ERES TAN PREPOTENTE que quieres ver a todos los hombres arrastr�ndose por ti.
	ACABAR�S SOLA......... TIEMPO AL TIEMPO....}''

``\textit{Nunca encontrar�s un hombre que te sorpote. Acabar�s sola, vieja, con gatos y escuchando techno}''

Otro insulto muy recurrente es ``mojigata'':


``\textit{?\#EnLaPedaNuncaFalta la mojigata persinada que con 3 tequilas encima es m�s f�cil que la tabla del 1. }''

``\textit{Qu� lindo que se le va cayendo la careta a la mojigata esa!!!!}''

La expresi�n ``mucho feminismo pero'' se utiliza para minusvalorar el papel del feminismo en la sociedad:

``\textit{?Pues menuda hija de mierda si tienes a tu padre cargando con TODAS las tareas del hogar. Mucho feminismo, pero de compartir las labores no te ense�� nada.}''

``\textit{Mucho feminismo pero Christian Gray se las hace mojar qu� tipo hdp este pibe jajaja}''

La expresi�n ``mujer al volante'' se utiliza habitualmente para minusvalorar la capacidad de las mujeres para realizar una tarea:


``\textit{por la prudencia con la que tomo la bajada creo que se trata de una mujer al volante }''

``\textit{No hay nada m�s peligroso que una mujer al volante con prisa . Nada, ni MALO De presidente es tan peligroso}''


Otra expresi�n que suele conllevar actitudes machistas ``las mujeres no deber�a'':


``\textit{Las mujeres no deber�an de tener ni voz ni voto, no est�n preparadas para mantener una familia menos un gobierno , son unas mantenidas}''

``\textit{Por eso es que las mujeres no deber�an tener derecho al voto}''

La frase ``A las mujeres hay que'' se utiliza habitualmente para discriminarlas o tratarlas de un modo distinto por su g�nero:


``\textit{Como dice mi pap�: ``a las mujeres hay que quererlas, no entenderlas''}''

``\textit{Hoy en d�a a las mujeres hay que tenerle un cuidado cabron. Acu�rdate que estamos en la era del chismin, la era mierdosa. Te da un abrazo y despu�s dice que le rozaste la teta. Wao! Pero si nos abrazamos! Obvio que las tetas feas esas van a rozar cabrona!}''


Otra expresi�n que suele conllevar actitudes machistas ``odio a las mujeres'':


``\textit{Ok. Con esto la loca se gano a todas las femininazis. Apuesto cualquier cosa, que salen las feministas a una marcha diciendo que Macri odia a las mujeres. En EEUU tambien fue campa�a con trump machista y su "odio a las mujeres". Asi se maneja la izquierda, campa�a: odio}''

``\textit{A lo que hemos llegado, lo unico que van a conseguir las feministas es el odio a las mujeres}''


Otra expresi�n que suele conllevar actitudes machistas ``las mujeres de hoy en dia'':

``\textit{Las mujeres de hoy en d�a son tan doble moral; no puedes decirle a una chica que trae el bot�n desabrochado de la blusa, por que se emperra y te tacha de pervertido, ah pero si no le dices te va peor. Tampoco es que sea un santo pero que no mamen. \#LaHubieraDejadoEnse�arTeta }''

``\textit{Se enojan las mujeres de hoy en d�a que uno les diga que parecen hombres, m�s insensibles que una piedra y m�s in�tiles en las actividades de a diario, puta no sean tan maletas prep�rense, por lo menos yo hice mi obra hoy ya se defiende en un mont�n de �reas de nada...}''

Por �ltimo, el insulto ``lagartona'' se utiliza de forma despectiva hacia las mujeres:

``\textit{Trata de nublar la verdad esta lagartona, aparte super fea}''

``\textit{No quiero depender de nadie. No soy una lagartona ni la zorrita de nadie, si no un hombre}''


\section{Generaci�n del corpus, ``\textit{Crawler}''}
\label{sec:Metric_Eval}

\noindent En este apartado, se especifica el proceso utilizado para extraer la informaci�n deseada de Twitter. En la fase de creaci�n del corpus se ha utilizado el API de Twitter a trav�s del paquete \textit{rtweet} disponible para el lenguaje de programaci�n R. 

Se ha desarrollado un \textit{Crawler} que recopila 100 tweets de cada t�rmino del la tabla 4.1, idealmente, ser�an 2900 tweets. De este modo, no se superar�a el l�mite diario del API de Twitter y se recopilar�a bastante informaci�n diaria. Al llegar a 15000 tweets para un t�rmino, se considera que hay suficiente informaci�n y se dejar� de buscar los tweets que lo contengan.

Al recolectar todos los tweets, se eligir�n aleatoriamente 150 tweets para cada t�rmino. Por tanto, ser� un requisito obtener, al menos, 150 tweets para cada t�rmino para poder ser considerado.

Para esta tarea, se ha estado recopilando informaci�n con el \textit{Crawler} durante las fechas 1/07/2018-31/12/2018, haciendo un total de 6 meses de informaci�n. Durante este periodo, el \textit{Crawler} recolect� un total de 181792 tweets para todos los t�rminos. Para cada \textit{tweet}, el API de Twitter permite acceder a 42 atributos distintos:

\begin{itemize}
\itemsep0em 
\item status\_id: identificador �nico del tweet.             
\item created\_at: fecha de creaci�n del tweet.      
\item user\_id: identificador �nico de usuario.              
\item screen\_name: alias que el usuario utiliza para identificarse.
\item text: mensaje del tweet.                  
\item source: dispositivo/cliente utilizado para publicar el mensaje.                
\item reply\_to\_status\_id: si es una respuesta, indica el id del tweet original.    
\item reply\_to\_user\_id: si es una respuesta, indica el id del usuario original.      
\item reply\_to\_screen\_name: si es una respuesta, indica el alias del usuario original.
\item is\_quote: indica si el tweet es citado.              
\item is\_retweet: indica si el tweet es un retweet.
\item favorite\_count: conteo aproximado del n�mero de favoritos.        
\item retweet\_count: conteo aproximado del n�mero de tweets.         
\item hashtags: hastags utilizados en el tweet.              
\item symbols: s�mbolos contenidos en el mensaje.               
\item urls\_url: URLs contenidas en el texto del tweet.              
\item urls\_t.co: URLs en formato acortado.             
\item urls\_expanded\_url: URLs en formato expandido.     
\item media\_url: URLs de los elementos subidos con el tweet.             
\item media\_t.co: URLs acortadas de los elementos subidos con el tweet.           
\item media\_expanded\_url: URLs expandidas de los elementos subidos con el tweet.  
\item media\_type: tipo de elemento subido junto al tweet (por ejemplo, foto).            
\item ext\_media\_url: URLs del elemento externo adjunto al tweet.         
\item ext\_media\_t.co: URLs acortada del elemento externo adjunto al tweet.                
\item ext\_media\_expanded\_url: URLs expandidas del elemento externo adjunto al tweet.         
\item ext\_media\_type: tipo del elemento externo adjunto al tweet.        
\item mentions\_user\_id: identificadores de otros usuarios nombrados en el tweet.      
\item mentions\_screen\_name: alias de otros usuarios nombrados en el tweet.  
\item lang: idioma del tweet.                  
\item quoted\_status\_id: identificador del tweet citado.       
\item quoted\_text: texto del tweet citado.           
\item retweet\_status\_id: si es un retweet, indica el identificador del tweet original.     
\item retweet\_text: texto del retweet original.
\item place\_url: URL que representa la localidad y aporta informaci�n adicional.            
\item place\_name: nombre de la localidad.            
\item place\_full\_name: nombre de la localidad con informaci�n a�adida (por ejemplo, provincia).       
\item place\_type: tipo de localidad (por ejemplo, ciudad).            
\item country: pa�s.               
\item country\_code: abreviatura del pa�s.          
\item geo\_coords: longitud y latitud del tweet.            
\item coords\_coords: longitud y latitud del tweet en distinto formato.         
\item bbox\_coords: un cuadro delimitador de coordenadas que encierra este lugar          
\item latitud: latitud del tweet.               
\item longitud: longitud del tweet.             
  
\end{itemize}


\subsection{Resultados de la creaci�n del corpus}
\label{sec:Resultados} 
\noindent A continuaci�n, se presentan las caracter�sticas y estructura principal del corpus generado. 

La tabla 4.2 representa el n�mero de tweets recopilado por t�rmino. Como se puede observar, se ha recopilado informaci�n para todos los t�rminos pero existen 4 de ellos para los que se han recopilado menos de 150 mensajes y que ser�n descartados por no disponer de suficiente informaci�n. Adem�s, es importante remarcar que la cantidad de informaci�n var�a notablemente seg�n los t�rminos, a partir del t�rmino 11 la cantidad de informaci�n por t�rmino se reduce en m�s de la mitad.

\begin{table}[]
	\centering
	\begin{tabular}{ll}
		T�rmino & N \\
		como una mujer & 15094 \\
		feminazi & 15093 \\
		a la cocina & 15087 \\
		zorra & 15086 \\
		loca del & 15084 \\
		como una nina & 15080 \\
		las feministas & 15076 \\
		ninata & 15032 \\
		en tus dias & 14190 \\
		a fregar & 14013 \\
		mojigata & 6008 \\
		marimacho & 5770 \\
		para ser mujer & 4693 \\
		nenaza & 4358 \\
		odio a las mujeres & 2749 \\
		lagartona & 2006 \\
		A las mujeres hay que & 1845 \\
		las mujeres no deberian & 1285 \\
		las mujeres de hoy en dia & 991 \\
		mujer al volante & 962 \\
		mucho feminismo pero & 852 \\
		mujer tenias que ser & 683 \\
		pareces una puta & 474 \\
		para ser chica & 180 \\
		acabaras sola & 50 \\
		hombre que te aguante & 37 \\
		obsesionada con el machismo & 8 \\
		pareces una fulana & 5 \\
		no ha probado un hombre & 1
	\end{tabular}
	\caption{N�mero de tweets por t�rmino}
	\label{my-label}
\end{table}

Como se ha comentado, el \textit{crawler} recolect� informaci�n desde el 01/07/2018 hasta 31/12/2018. En la figura 4.1 se puede observar el n�mero de tweets recopilados por d�a. 
\begin{center}
	\includegraphics[scale=0.8,keepaspectratio]{imagenes/tweets_dia.png} %[width=4cm,,keepaspectratio]
	\captionof{figure}{Tweets recopilados diariamente}	
\end{center}
Como se puede observar, el n�mero de tweets recopilados diariamente dista mucho de los 2400 que podr�an generarse si se encontraran los 100 tweets objetivo para cada t�rmino. Adem�s, a partir del d�a 29/11/2018 se reducen los tweets recopilados al encontrar los 15000 tweets objetivo para alguno de los t�rminos. De todos estos tweets, se puede observar en la figura 4.2 como Espa�a es el pa�s que m�s genera.

\begin{center}
	\includegraphics[scale=0.8,keepaspectratio]{imagenes/tweets_pais.png} %[width=4cm,,keepaspectratio]
	\captionof{figure}{Tweets recopilados diariamente}	
\end{center}

Otra caracter�stica interesante es el uso de \textit{hastags} en los tweets, en total, se han recopilado 15218 hastags. En la figura 4.3 se pueden observar los t�rminos m�s utilizados. El \textit{hastags} m�s utilizado coincide con uno de los t�rminos utilizados para la recopilaci�n de los datos por el \textit{crawler}. El resto parecen estar relacionados con el programa televisivo ``Gran Hermano''. En la figura 4.4 se puede observar como Espa�a es el pa�s que m�s \textit{hastags} utiliza.

\begin{center}
	\includegraphics[scale=0.8,keepaspectratio]{imagenes/hastags.png} %[width=4cm,,keepaspectratio]
	\captionof{figure}{Tweets recopilados diariamente}	
\end{center}

\begin{center}
	\includegraphics[scale=0.8,keepaspectratio]{imagenes/hastags_tweet_pais.png} %[width=4cm,,keepaspectratio]
	\captionof{figure}{Tweets recopilados diariamente}	
\end{center}


Para identificar si los tweets utilizan una gram�tica adecuada, se ha observado las palabras OOV (\textit{out of vocabulary}) utilizando un diccionario espa�ol. En la figura 4.5 se muestra como los Estados Unidos es el pa�s con m�s OOV por tweet.

\begin{center}
	\includegraphics[scale=0.8,keepaspectratio]{imagenes/oov.png} %[width=4cm,,keepaspectratio]
	\captionof{figure}{Tweets recopilados diariamente}	
\end{center}

Por �ltimo, se han estudiado los enlaces por tweets. En total, se han recopilado 25539 enlaces para todos los mensajes. En la figura 4.6 se observa como Espa�a es el pa�s que m�s enlaces por tweet utiliza.

\begin{center}
	\includegraphics[scale=0.8,keepaspectratio]{imagenes/enlaces_tweet_pais.png} %[width=4cm,,keepaspectratio]
	\captionof{figure}{Tweets recopilados diariamente}	
\end{center}

\section{Etiquetado del corpus}
\label{sec:Col_Eval}

\noindent Tras la ejecuci�n del \textit{crawler}, es necesario componer el corpus a etiquetar para poder aplicar un algoritmo de aprendizaje supervisado.

Para realizar el corpus, se han utilizado las 24 etiquetas con m�s de 150 tweets que se puede observar en la tabla 4.2. Para cada t�rmino, se realiza un muestreo aleatorio de 150 tweets haciendo un total de 3600 tweets para ser etiquetados. Para todos los t�rminos, se ha comprobado que la separaci�n temporal entre sus tweets es m�s de un d�a, de este modo, se evitan mensajes que puedan ser conversaciones dentro del mismo d�a.

El objetivo principal del presente trabajo la detecci�n del lenguaje machista en espa�ol en la red de microblogging Twitter. Para ello, se propone la clasificaci�n del contenido procedente de esta red social en tres categor�as distintas: machista, no machista y dudoso. A continuaci�n, se definen las etiquetas propuestas:

\begin{itemize}
	\itemsep0em 
	\item MACHISTA: tweets que contienen connotaciones machistas/ofensivas hacia las mujeres. Los tweets que pertenecen a esta categor�a deben de contener actitudes discriminatorias o hacia las mujeres por su sexo. Ejemplo:
	``\textit{@EmanuelGPA Lo ir�nico es que lo dice una mujer, que ``naturalmente'' deber�a callarse y dedicarse a la cocina, limpiar y criar hijos}''
	En este ejemplo, se infravalora a la mujer de un modo expreso.
	             
	\item NO\_MACHISTA: tweets que no contienen connotaciones machistas. Ejemplo:
	``\textit{�Por qu� cuando ven a una chica con pelo corto piensan que es bi, torta o marimacho? ...Capaz es solo un look o que el pelo largo le da mucho calor... Y si es...�Cual es el problema?}''
	Dentro de esta categor�a se clasifican tambi�n mensajes xen�fobos y ofensivos en general pero que no discriminan a las mujeres por su sexo. Ejemplo: 
	``\textit{@kenia773 @LuisCarlos POR CIERTO, EN TU FOTO DE PERFIL SE PUEDE OBSERVAR QUE ERES BASTANTE VARONIL, AS� QUE SI NO ERES MARIMACHO, EMPIEZA A SERLO}''
	\item DUDOSO: tweets que, dependiendo del contexto, no presente en el tweet, podr�an ser machistas (si el fragmento ofensivo se refiriese a mujeres). Ejemplo:        
	``\textit{@hazteoir @PSOE M�s vale que se marche a fregar!}''
	Para poder considerar el tweet machista es necesario una referencia expresa a las mujeres, pese a que en el texto se pueda realizar una comparaci�n con una mujer mediante una expresi�n machista.                  
	
\end{itemize}

El proceso de etiquetado ha sido llevado a cabo por 3 anotadores que han analizado los 3600 tweets que conforman el corpus para asignar una de las 3 etiquetas propuestas. Para ello, se ha desarrollado una gu�a de etiquetado en la que se explican las etiquetas, se proponen ejemplos de cada una de ellas y se establecen las normas de entrega. \todo{A�adir la gu�a como ANEXO}. El etiquetado se llev� a cabo entre los meses de Enero y Marzo de 2019 mediante la herramienta ``Google Sheets'' un entorno de trabajo colaborativo online que permite la creaci�n de hojas de c�lculo.

Durante el anotado del corpus, cada uno de los 3 etiquetadores propone una clase de las 3 disponibles para cada tweets. Al t�rmino del etiquetado, se decidir� por votaci�n de los 3 anotadores la etiqueta final. En caso de desacuerdo total entre los 3, un cuarto etiquetador decidir� la clase final para el tweet.


\subsection{Dificultades encontradas en el etiquetado del corpus}
\label{sec:Col_Eval}
3.2-Problemas encontrados: ambiguedad (tweets en los que tuvimos desacuerdo total), citaciones de expresiones machistas, redefinici�n de las reglas de etiquetado por fallo del kappa en el 20%

403: @mccallsoul\_ ya me joder�a si pareces una puta reina con todo lo que te pones ??
573: @tonifreixa La zorra guardando las gallinas. �� Que se encargue Rosell �� Bueno..., cuando salga de la c�rcel. Cinismo en grado m�ximo.
770: @StopHembristas Como si las feministas no hicieran lo mismo con personas que no comportan su opini�n
820: Hermano, la regla no le baja solo a las feministas xd https://t.co/OYNcUjO9ql
853: Ella aguantaa como una mujer ....Pero llora como una ni�a...?????? https://t.co/pxbE2CaZ5D
878: pareces una puta vieja no mames wey
951: @rhm1947 Pareces una puta chavista
958: @marijopellicer @radchiaru @\_lxuli Jajajaja si no sabes cuanto odio a las mujeres tengo ?????? por favor no veas enemigos donde no los hay.
1014: @CesarRM22 Por qu� saben que las mujeres de hoy en d�a son m�s de lo que merecen ???
1018: @\_ohara\_ Zorra pero no tuya mi amor :,*
1033: Cuando subes a tu amiga la lagartona al Uber porque ya andaba malacopeando https://t.co/DcnK5ZGuL4
1397: ?Bueno, para ser chica y tan joven tiene buena estatura.?
1400: Sin querer me chut� como un importante del work le hablaba super sucio a la m�s - LA MAS - mojigata de aqu� no cabe duda que las apariencias enga�an y la m�s calladita es la m�s ?? ??
1539: @todonoticias TN pareces una puta barata SOS el �nico canal q habla de esas idioteces a y canal trece que son la misma miarda junto con Clar�n porque no hablas de la realidad de este gobierno de miarda que cada vez Unde m�s al pa�s
2141: Pareces una puta despechada @peladoalmeyda
2190: Mucho feminismo pero a la primera de cambio..... https://t.co/y2McecsgcT
2865: @danielbroxlr Donde vas zorra asquerosa
2876: @silvia0907 Tu eres Tonta, ponte a fregar anda
3124: ?Mujer ten�as que ser?... cu�ntas veces habremos escuchado esta frase? Compartimos una de nuestras experiencias, l�ela y no dudes en compartir la tuya.. https://t.co/2xhBHfcpRo https://t.co/7Dsl4AZq6V
3236: Mi abuela esta mandando a yesla a fregar,a guardarlos a todo juju.


%%%%%%%%%%%%%%%%%%%%%%%%%%%%%%%%%%%%%%%%%%%%%%%
Hola,

Os escribo porque estamos teniendo una duda en el etiquetado para un tipo de tweets.

En concreto, tenemos dudas en aquellos tweets en los que la persona que escribe el mensaje cita un contenido machista (en ocasiones con el que est� en desacuerdo). Por ejemplo, un tweet que me indicasteis que era machista:

?Pareces una puta con ese pantal�n? - Mi hermano de 13 cuando me vio con un pantal�n de cuero

Jhoni pone esto como no machista, cuando para nosotros es claramente machista.


En este caso creo que s� que es machista porque, aunque se tenga contexto suficiente para afirmar que la frase machista es citada por la persona que escribe el tweet, no se comenta nada sobre su contenido. Por ejemplo, tenemos otros casos que me han estado consultando los etiquetadores:

Cada vez m�s a menudo (todos lo d�as) mi padre me dice que las mujeres no deber�an recibir premios, trabajar en puestos superiores, que son putas, y que deben quedarse en casa y servir al hombre y criar hijos.

@localgothgirI Me pasa todo el tiempo, incluso, cuando me va mal, siempre recibo insultos y comentarios como ten�a que ser mujer las mujeres no deber�an jugar este juego\". A veces se me quitan las ganas de seguir jugando

@mariarodd17 T�a ella diciendo cosas tipo: es que las mujeres de hoy en d�a son muy sueltas y son unas acosadoras, est�s ahora van preparadas con to y van a los hombres dici�ndoles vamos a pasar un rato que ni te cobro y ellos los pobres si tienen pareja tienen que ir preparados para no caer

En los anteriores, yo creo que hay contexto suficiente para entender que se cita un contenido machista como ejemplo para apoyar su desacuerdo con ese tipo de lenguaje. Por esta raz�n, creo que se tratar�a de tweets no machistas.









Buenas Francisco,

Hemos estado hablando sobre este tema Laura y yo pensamos que hay que incluir todo, tanto citas, como chistes, como incluso cr�ticas al machismo con cita (en plan, ??Mujer ten�as que ser?, odio este tipo de respuestas en los hombres?). El objetivo que tenemos en mente con este trabajo es identificar todo tipo de expresi�n machista, e incluso por, as� decirlo, determinar mediante eso la cantidad de machismo existente en las redes sociales.

Si nos centramos solo en las expresiones machistas realizadas para ofender, excluimos, por ejemplo, todas las vivencias machistas que una mujer puede relatar y que ha vivido. Incluso, los chistes machistas, aunque hechos en ciertos casos para no ofender, son machistas, por lo que deben ser considerados.

Quiz�, una vez realizado esto, y en estudios posteriores independientemente del TFM, estar�a bien discriminar a posteriori lo que es machista y est� escrito para ofender, de lo que es machista y no es para ofender, si no para describir una vivencia, etc.

Como lo ves?

Un saludo.






%%%%%%%%%%%%%%%%%%%%%%%%%%%%%%%%%%%%%%%%%%%%%%%%%%%



porque la herramienta que me pasaste me estaba dando un problema porque solo pod�an ser ficheros con 3 columnas. La media del coeficiente para los 3 pares me sale de 0.51. Seg�n esta referencia  es un acuerdo moderado entre los 3 (por lo que parece vamos en buen camino).







Utilizando esta tabla, he estado viendo los patrones y desacuerdos m�s llamativos. He sacado las siguientes conclusiones:



El etiquetador 3 ha utilizado mucho m�s la etiqueta MACHISTA. He estado revisando y hay bastantes en los que los criterios no est�n del todo bien aplicados. Se confunde muchas veces xenofobia con machismo.
El etiquetador 1 confunde DUDOSO con MACHISTA. He estado viendo las etiquetas para este caso y se utiliza dudoso en casos en los que hay un contexto claro para etiquetar como MACHISTA.








a me han pasado los etiquetadores sus nuevas etiquetas para los 720 tweets. En este caso los resultados del acuerdo son mucho mejores:

Kappa Etiquetadores 1 - 2: 0.7617779499346342
Tasa de coincidencia etiquetadores 1 - 2 (%): 88.6111111111
Kappa Etiquetadores 1 - 3: 0.7780535422117186
Tasa de coincidencia etiquetadores 1 - 3 (%): 89.1666666667
Kappa Etiquetadores 2 - 3: 0.7391428980937366
Tasa de coincidencia etiquetadores 2 - 3 (%): 87.3611111111
Tasa de coincidencia etiquetadores 1 - 2 - 3 (%): 82.7777777778
Media Kappa Etiquetadores 1 - 2 - 3: 0.7596581300800298

A la vista de estos resultados y habiendo aclarado las dudas, vamos a continuar con el etiquetado durante 2 semanas. El d�a 23 volver� a comprobar que tal va el acuerdo para monitorizar un poco el proceso. 




\subsection{Resultados del etiquetado del corpus}
\label{sec:Col_Eval}

3.4 - Kappa alcanzado, acuerdo entre anotadores (meter m�tricas de Excel).

Ya hemos terminado el etiquetado los 3 anotadores. El kappa final es de 0.75 aproximadamente (media por parejas), por lo que parece que hay un buen acuerdo. Al correo, os adjunto los tweets en los que hemos tenido desacuerdo entre los 3 y es necesario que nos deis una etiqueta. Cuando lo tenga, probar� el clasificador con las etiquetas definitivas.

\begin{center}
	\includegraphics[scale=0.8,keepaspectratio]{imagenes/etiquetadores_final.png} %[width=4cm,,keepaspectratio]
	\captionof{figure}{Tweets recopilados diariamente}	
\end{center}


\begin{center}
	\includegraphics[scale=0.8,keepaspectratio]{imagenes/porcentaje_acuerdo.png} %[width=4cm,,keepaspectratio]
	\captionof{figure}{Tweets recopilados diariamente}	
\end{center}


5.2-An�lisis de la exploraci�n del corpus de spyder



         index  categoria
0  NO\_MACHISTA  61.361111
1     MACHISTA  31.083333
2       DUDOSO   7.555556

NO\_MACHISTA    2209
MACHISTA       1119
DUDOSO          272


\begin{center}
	\includegraphics[scale=0.8,keepaspectratio]{imagenes/clase_distribucion.png} %[width=4cm,,keepaspectratio]
	\captionof{figure}{Tweets recopilados diariamente}	
\end{center}




\begin{center}
	\includegraphics[scale=0.8,keepaspectratio]{imagenes/numericas_relevantes_pairplot.png} %[width=4cm,,keepaspectratio]
	\captionof{figure}{Tweets recopilados diariamente}	
\end{center}



\begin{center}
	\includegraphics[scale=0.3,keepaspectratio]{imagenes/categoricas_pairplot.png} %[width=4cm,,keepaspectratio]
	\captionof{figure}{Tweets recopilados diariamente}	
\end{center}


%%%%%%%%%%%%%%%%%%%%%%%%%%%%%%%%%%%%%%%%%%%%%%%%%%%
%%% SISTEMA
%%%%%%%%%%%%%%%%%%%%%%%%%%%%%%%%%%%%%%%%%%%%%%%%%%%

\chapter{Sistema/M�todo/Caso de Estudio propuesto}
\fancyhead[RE]{\textsc{CAP�TULO} \thechapter. Sistema/M�todo/Caso de Estudio propuesto}
\label{ch:Sistema_Metodo_Caso_de_Estudio}

\noindent En este cap�tulo se describe en profundidad el sistema/m�todo o caso de estudio propuesto. La organizaci�n de este cap�tulo depender� sustancialmente del trabajo abordado.


\section{Ejemplo secci�n}
\label{sec:Ejemplo_seccion}

\subsection{Ejemplo subsecci�n}
\label{sec:Ejemplo_subSeccion} 
%%%%%%%%%%%%%%%%%%%%%%%%%%%%%%%%%%%%%%%%%%%%%%%%%%%
%%% EVALUACION
%%%%%%%%%%%%%%%%%%%%%%%%%%%%%%%%%%%%%%%%%%%%%%%%%%%

\chapter{Evaluaci�n y discusi�n}
\fancyhead[RE]{\textsc{CAP�TULO} \thechapter. Evaluaci�n}
\label{ch:Evaluacion}

\noindent En el siguiente cap�tulo se presentan los procedimientos de evaluaci�n as� como los resultados de los experimentos realizados. Adem�s, se realizar� una comparativa y discusi�n de los distintos resultados obtenidos. Para evaluar el sistema propuesto, se definen dos experimentos seg�n el procedimiento seguido para realizar la evaluaci�n del sistema. En el primero, se reservan una parte de los datos para realizar una b�squeda de los hiperpar�metros �ptimos para cada algoritmo de clasificaci�n mientras que en el segundo se emplean los par�metros por defecto para evitar el sobreajuste. Con estos experimentos, se pretende evaluar el rendimiento del sistema propuesto para detecci�n del machismo en redes sociales. Asimismo, se evaluar� el efecto que tiene en el sistema el desbalanceo existente en la clase del conjunto de datos.

\section{Metodolog�a de evaluaci�n}
\label{sec:Met_Eval}

\subsection{M�tricas de evaluaci�n}
\label{sec:Col_Eval}

\noindent Para la evaluaci�n de los resultados en clasificaci�n textual o de documentos se utiliza com�nmente la matriz de confusi�n. Se trata de una una herramienta que representa en cada columna el n�mero de predicciones de cada clase, mientras que cada fila representa a las instancias en la clase real. En la siguiente imagen se presenta un esquema de la matriz de confusi�n:

\begin{center}
	\includegraphics[width=0.5\textwidth]{imagenes/confusion_matrix_1.png} %[width=4cm,,keepaspectratio]
	\captionof{figure}{Matriz de confusi�n}	
\end{center}

Esta tabla est� formada por verdaderos positivos, verdaderos negativos, falsos positivos y falsos negativos. De este modo, si un documento es clasificado por el sistema autom�tico en la misma categor�a que la clasificaci�n manual, se considerar� como un verdadero positivo o negativo (\textit{True Positive}, TP o \textit{True Negative}, TN), mientras que si el documento es clasificado por el sistema con una categor�a diferente, se estar� ante un falso negativo o falso positivo (\textit{False Positive}, FP o \textit{False Negative}, FN).Utilizando estos cuatro componentes se calculas las medidas principales para evaluar los resultados:

\begin{itemize}
	\itemsep0em 
	\item Tasa de acierto o exactitud (\textit{accuracy}): representa el porcentaje de aciertos en relaci�n a todos los documentos clasificados.
	\[\textit{Accuracy}=\frac{TP+TN}{TP+FP+TN+FN}\]
	\item Precisi�n: representa la fracci�n de asignaciones correctas frente al total de asignaciones positivas realizadas para esa clase. Es decir, realiza una medida de la tasa de acierto para un valor de la clase.
	\[Precision=\frac{TP}{TP+FP}\]
	\item Cobertura (\textit{recall}): representa la fracci�n de asignaciones positivas respecto al conjunto real de elementos pertenecientes a la clase. Es decir, realiza una medida de la capacidad que tiene el clasificador de detectar elementos de esa clase.
	\[Cobertura=\frac{TP}{TP+FN}\]
	\item Medida-F: combina las medidas de precision y cobertura.
	\[Medida-F=\frac{2xprecisionxcobertura}{precision+cobertura}\]
\end{itemize}


\subsection{Colecci�n de evaluaci�n}
\label{sec:Col_Eval}

\noindent Las colecciones de evaluaci�n son conjuntos de datos etiquetados con informaci�n relevante para la tarea para la cual han sido desarrollados. En este caso, las colecciones de evaluaci�n para clasificaci�n de documentos est�n compuestas por textos, ya sean oraciones, p�rrafos o documentos completos, de distinta naturaleza y que est�n etiquetados con categor�as concretas. Por ejemplo, para el presente trabajo, existen tres valores posibles para esta categor�a: "MACHISTA", "NO\_MACHISTA" y "DUDOSO".

Estos conjuntos de evaluaci�n permiten intuir el rendimiento de los sistemas de clasificaci�n y compararlo con el de otros sistemas. Asimismo, en los sistemas de clasificaci�n supervisados son clave para poder ser entrenados utilizando un subconjunto de la colecci�n.

Para el presente trabajo, se utilizar� como conjunto de evaluaci�n del sistema de clasificaci�n de contenido machista el corpus presentado en el cap�tulo 4. Se trata de un corpus compuesto por 3600 \textit{tweets} recopilados mediante el uso de expresiones que pueden conllevar actitudes machistas. 

Para recuperar esta informaci�n se utilizaron los siguientes t�rminos: ``feminazi", ``loca del", ``a la cocina", zorra, ``como una ni�a", ``las feministas", ni�ata, ``como una mujer", ``en tus d�as", ``a fregar", mojigata, marimacho, nenaza, ``para ser mujer", ``odio a las mujeres", lagartona, ``A las mujeres hay que", ``las mujeres no deber�an", ``las mujeres de hoy en d�a", ``mujer al volante", ``mujer ten�as que ser", ``mucho feminismo pero", ``pareces una puta", ``para ser chica". De este modo, se recopilaron todos los mensajes escritos en la red social que contuvieran estos t�rminos durante las fechas 1/07/2018-31/12/2018.

El prop�sito principal de este corpus es la obtenci�n de texto con alto contenido machista as� como expresiones que, a�n pudiendo ser machistas, no lo sean en ese contexto. De este modo, se pretende obtener un conjunto rico en el uso de expresiones que pueden conllevar actitudes machistas en diferentes contextos. Para conseguir captar estos matices, el corpus est� etiquetado con las categor�as "MACHISTA", "NO\_MACHISTA" y "DUDOSO". 

\subsection{L�neas base (\textit{baseline})}
\label{sec:Col_Eval}

\noindent Como se ha ido introduciendo en el cap�tulo 2, las referencias en el campo de detecci�n del machismo son muy reducidas y, por tanto, es complejo encontrar alg�n trabajo comparable con el sistema desarrollado. Es por esto que en este trabajo se han desarrollado dos l�neas base con las que comparar los resultados obtenidos por el sistema dise�ado. La primera de ellas plantea un sistema de clasificaci�n basado en una regresi�n log�stica sobre los atributos \textit{tf-idf} utilizando los unigramas de cada documento. De este modo, se plantea un sistema sencillo pero pudiendo ser, en ocasiones, mucho m�s efectivo que otras aproximaciones m�s complejas que utilizan bi-gramas o categor�as gramaticales de los t�rminos, por tanto, se trata de un \textit{baseline} dif�cil de batir.

La segunda l�nea base est� basada en un clasificador sencillo que predice siempre la clase mayoritaria. En este caso, como se puede observar en la tabla 4.7, la clase mayoritaria ser�a "NO\_MACHISTA" y, por tanto, este sistema clasificar�a todos los registros de entrada con este valor de clase. La intenci�n de esta l�nea base es comparar los resultados del sistema con otro hipot�tico no informado que no es capaz de "aprender" ning�n patr�n del conjunto de datos de entrenamiento.

\subsection{Experimento 1: B�squeda de hiperpar�metros mediante la optimizaci�n de la medida F1}
\label{sec:Col_Eval}

\noindent El primer experimento realizado trata de configurar cada algoritmo de clasificaci�n para la tarea especifica que van a desarrollar. Como se introdujo en el cap�tulo 5 para la tarea de clasificaci�n se emplean 3 algoritmos distintos disponibles en ``scikit-learn'': Regresi�n log�stica, Random Forest y SVM. En este primer experimento, se realiza una b�squeda para los siguientes par�metros:

\begin{itemize}
	\itemsep0em 
	\item Regresi�n log�stica: C = [1, 10], class\_weight' = [None, 'balanced'].
	\item Random Forest: n\_estimators" = [250, 450], bootstrap' = (True, False), max\_depth'= [None, 30].
	\item SVM: C = [1, 10, 100, 10000], gamma = [0.001, 0.1, 0.6, 'auto'], kernel = 'rbf'.
\end{itemize}

Para ello, se sigue el procedimiento presentado en la figura 6.2 de forma iterativa. En el primer paso, se realizan diez reparticiones aleatorias en dos conjuntos de datos: entrenamiento (training) y testeo (testing). Para el conjunto de training, se reservan el 30\% de los datos y para el test, el resto. Para cada uno de los diez repartos, se utiliza el conjunto de training para la b�squeda de hiperpar�metros y el testing para evaluar los resultados con los par�metros encontrados.

Para la b�squeda de par�metros, se realiza una validaci�n cruzada (\textit{cross validation}) con cinco grupos. En este proceso, se realizan cinco conjuntos (que coincidir�an con los 5 "splits" de la figura) con los datos de entrenamiento y, posteriormente, se realiza el entrenamiento en cuatro de ellos y el testeo en el grupo restante. Este proceso permite obtener los par�metros que mejor han funcionado en el proceso seg�n el valor de la medida F1. Este proceso se repetir�a para cada uno de los 5 grupos.

En la siguiente etapa del proceso se utilizar�a el segundo conjunto de datos reservado para el testeo que coincidir�a con el 70\% de todo el corpus para realizar la evaluaci�n final. Utilizando los par�metros de entrada obtenidos en la etapa anterior, se realizar�a una validaci�n cruzada con 10 grupos del conjunto de testing. De nuevo, en este proceso se realizar�a un reparto en diez grupos, donde nueve de ellos ser�n utilizados para el entrenamiento y el grupo restante para el testeo, repitiendo el proceso diez veces, una por grupo. Estas dos etapas, se repiten para los diez repartos indicados al inicio.

Este experimento permite medir el resultado de un sistema dise�ado especificamente para esta tarea pues la configuraci�n de los algoritmos de clasificaci�n se realiza seg�n los datos del corpus. Adem�s, al realizar diez iteraciones para el proceso, la varianza de los resultados se reduce y son menos dependientes del tipo de datos con el que se ha entrenado. La desventaja principal de este m�todo es que es necesario reservar un conjunto de datos para la b�squeda de par�metros y se reduce la informaci�n de la que dispondr� el sistema de clasificaci�n definitivo que realizar� la predicci�n.

\begin{center}
	\includegraphics[scale=0.4,keepaspectratio]{imagenes/grid_search_cross_validation.png} %[width=4cm,,keepaspectratio]
	\captionof{figure}{B�squeda de hiperpar�metros mediante la optimizaci�n de la medida F1}
\end{center}

\subsection{Experimento 2: Cross validation con par�metros por defecto}
\label{sec:Col_Eval}

\noindent El segundo experimento consiste en una �nica validaci�n para todo el corpus utilizando los par�metros por defecto para todos los algoritmos de clasificaci�n utilizados.

En este caso, se ha optado por una validaci�n cruzada con diez grupos. En la figura 6.3 se presenta un ejemplo equivalente para cinco grupos. En este procedimiento, se realizar�a una divisi�n del conjunto en 10 grupos del mismo tama�o del corpus y, de forma iterativa, se utilizar�n nueve de ellos para el entrenamiento y el grupo restante para el testeo.

Este m�todo permite evaluar un sistema m�s general cuyos par�metros de configuraci�n no est�n dise�ados para los datos de entrenamiento de los que se disponen. Esto permite mejorar la capacidad de generalizaci�n del sistema y evitar un posible sobreajuste.

\begin{center}
	\includegraphics[scale=0.55,keepaspectratio]{imagenes/cross_validation_5.png} %[width=4cm,,keepaspectratio]
	\captionof{figure}{Validaci�n cruzada k=5}
\end{center}

\section{Resultados experimento 1}
\label{sec:Metric_Eval}

La tabla 6.1 muestra los resultados promedio de exactitud (\textit{Accuracy}), medida F1, cobertura (\textit{Recall}) y precisi�n. Con el m�todo de evaluaci�n descrito para el primer experimento, el algoritmo \textit{Random Forest} alcanza una mayor tasa de acierto y precisi�n mientras que la regresi�n log�stica alcanza los mejores resultados para la medida F1 y \textit{recall}. 

En relaci�n a la comparaci�n de los resultados obtenidos por el m�todo con las dos l�neas bases, en este caso las diferencias con respecto a la aproximaci�n basada en unigramas son de unos cuatro puntos porcentuales para cada medida. Por tanto, el sistema mejora esta primera aproximaci�n en todas las medidas pero, como se preve�a, la l�nea base ya es un buen punto de partida del sistema.

En el caso de la l�nea base basada en el clasificador sencillo de la clase mayoritaria, si se pueden observar grandes diferencias en las m�tricas de calidad. Esto indica, que cualquiera de las soluciones propuestas ser� mucho m�s adecuada que un clasificador basado en una �nica regla sencilla.

% Please add the following required packages to your document preamble:
% \usepackage{booktabs}
\begin{table}[]
	\centering
	\begin{tabular}{@{}lllll@{}}
		\toprule
		& \textbf{Accuracy} & \textbf{F1} & \textbf{Recall} & \textbf{Precision} \\ \midrule
		\textbf{Baseline (tf-idf)} & 0.68 & 0.59 & 0.62 & 0.59 \\
		\textbf{Baseline} & 0.61 & 0.2 & 0.3 & 0.24 \\
		\textbf{LR} & 0.7 & \textbf{0.62} & \textbf{0.64} & 0.62 \\
		\textbf{RF} & \textbf{0.72} & 0.6 & 0.57 & \textbf{0.67} \\
		\textbf{SVM} & 0.7 & 0.61 & 0.63 & 0.61 \\ \bottomrule
	\end{tabular}
	\caption{Resultados experimento 1}
	\label{tab:my-table}
\end{table}


\section{Resultados experimento 2}
\label{sec:Col_Eval}

La tabla 6.2 muestra los resultados promedio de exactitud (\textit{Accuracy}), medida F1, cobertura (\textit{Recall}) y precisi�n. Para este segundo experimento, el algoritmo RF consigue de nuevo la mejor tasa de acierto y precisi�n, mientras que en este caso es el algoritmo SVM el que obtiene los mejores resultados en cuanto a la medida F1 y el \textit{recall}. Sin embargo, es importante destacar que el comportamiento de los tres clasificadores es bastante similar en l�neas generales.

En relaci�n con el m�todo de evaluaci�n propuesto en el experimento 1, se puede observar una peque�a mejorar�a de un punto para cada medida de calidad.

En relaci�n a la comparaci�n de los resultados obtenidos por el m�todo con las dos l�neas bases, de nuevo, las diferencias con respecto a la aproximaci�n basada en unigramas son de unos cuatro puntos porcentuales para cada medida. 

En el caso de la l�nea base basada en el clasificador sencillo de la clase mayoritaria, se acrecientan las diferencias pues este segundo m�todo de evaluaci�n parece mejorar, en l�neas generales, los resultados del primer m�todo.

% Please add the following required packages to your document preamble:
% \usepackage{booktabs}
\begin{table}[]
	\centering
	\begin{tabular}{@{}lllll@{}}
		\toprule
		& \textbf{Accuracy} & \textbf{F1} & \textbf{Recall} & \textbf{Precision} \\ \midrule
		\textbf{Baseline (tf-idf)} & 0.69 & 0.58 & 0.56 & 0.62 \\
		\textbf{Baseline} & 0.61 & 0.2 & 0.3 & 0.24 \\
		\textbf{LR} & 0.72 & \textbf{0.63} & 0.62 & 0.65 \\
		\textbf{RF} & \textbf{0.73} & 0.61 & 0.58 & \textbf{0.68} \\
		\textbf{SVM} & 0.72 & \textbf{0.63} & \textbf{0.64} & 0.63 \\ \bottomrule
	\end{tabular}
	\caption{Resultados experimento 2}
	\label{tab:my-table}
\end{table}

\section{Efecto del desbalanceo de la clase}
\label{sec:Resultados}




%%%%%%%%%%%%%%%%%%%%%%%%%%%%%%%%%%%%%%%%%%%%%%%%%%%
%%% CONCLUSIONES Y TRABAJO FUTURO
%%%%%%%%%%%%%%%%%%%%%%%%%%%%%%%%%%%%%%%%%%%%%%%%%%%

\chapter{Conclusiones y trabajo futuro}
\fancyhead[RE]{\textsc{CAP�TULO} \thechapter. Conclusiones y trabajo futuro}
\label{ch:Conclusiones y trabajo futuro}

\noindent Este cap�tulo recopila las diferentes conclusiones extra�das del trabajo realizado, y propone algunas l�neas de trabajo futuro. Las siguientes secciones son las que suelen contener este tipo de cap�tulos, aunque pueden variar dependiendo del tema del trabajo.

\section{Conclusiones}
\label{sec:Conclu}


\section{Trabajo futuro}
\label{sec:Trab_Fut} 

%%%%%%%%%%%%%%%%%%%%%%%%%%%%%%%%%%%%%%%%%%%%%%%%%%%
%%% BIBLIOGRAF�A
%%%%%%%%%%%%%%%%%%%%%%%%%%%%%%%%%%%%%%%%%%%%%%%%%%%
\bibliographystyle{fullname_esp}
\cleardoublepage
\addcontentsline{toc}{chapter}{Bibliograf�a}
\chapter*{Bibliograf�a}
\fancyhead[RE]{BIBLIOGRAF�A}
\fancyhead[LO]{BIBLIOGRAF�A}
\bibliography{Bibliografia}


\newpage
\newpage

%%%%%%%%%%%%%%%%%%%%%%%%%%%%%%%%%%%%%%%%%%%%%%%%%%%
%%% APENDICES SI PROCEDE (COMENTAR)
%%%%%%%%%%%%%%%%%%%%%%%%%%%%%%%%%%%%%%%%%%%%%%%%%%%
%\appendix
%\chapter{Publicaciones}
%\label{ch:publicaciones}
%\fancyhead[RE]{\textsc{AP�NDICE} \thechapter. Publicaciones}
%\fancyhead[LO]{\textsc{AP�NDICE} \thechapter. Publicaciones}

%Publicaciones derivadas del trabajo realizado.

\appendix
\chapter{Gu�a de anotaci�n}
\includepdf[pages={1-7}]{Guia_anotacion.pdf}

%%%%%%%%%%%%%%%%%%%%%%%%%%%%%%%%%%%%%%%%%%%%%%%%%%%
%%% lISTA DE TAREAS PENDIENTES.
%%%%%%%%%%%%%%%%%%%%%%%%%%%%%%%%%%%%%%%%%%%%%%%%%%%
\newpage
\newpage
%Eliminar si no hay todo's
\listoftodos


\end{document}

%%%%%%%%%%%%%%%%%%%%%%%%%%%%%%%%%%%%%%%%%%%%%%%%%%%
%%% SCRIPT FINAL DE TODO'S O TAREAS PENDIENTES.
%%%%%%%%%%%%%%%%%%%%%%%%%%%%%%%%%%%%%%%%%%%%%%%%%%%
% Make the margin par
\marginpar{%
    \begin{tikzpicture}[remember picture]%
        \draw node[notestyle] (inNote)
                 {#1};%
    \end{tikzpicture}%
}%
%
\begin{tikzpicture}[remember picture, overlay]%
    \draw[connectstyle]
        ([yshift=-0.2cm] inText)
            -| ([xshift=-0.2cm] inNote.west)
            -| (inNote.west);
\end{tikzpicture}%

}% 