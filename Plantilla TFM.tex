%%%%%%%%%%%%%%%%%%%%%%%%%%%%%%%%%%%%%%%%%%%%%%%%%%%%%%%%
%%% Plantilla desarrollada por Jorge Carrillo de Albornoz
%%% para los trabajos de Fin de Master del M�ster en
%%% Lenguajes y Sistemas Inform�ticos de la UNED
%%% Cualquier duda o comentario: jcalbornoz@lsi.uned.es
%%%%%%%%%%%%%%%%%%%%%%%%%%%%%%%%%%%%%%%%%%%%%%%%%%%%%%%%

\documentclass[a4paper,11pt,twoside,openright,spanish]{book}

\usepackage{latexsym}
\usepackage[latin1]{inputenc}
\usepackage[activeacute,spanish]{babel}
\frenchspacing
\usepackage{eucal}
\usepackage{multirow}
\usepackage{fancyhdr}
\usepackage[final,pdftex]{graphicx}
\usepackage{subfigure}
\usepackage{amsmath}
\usepackage{fullname_esp}
\usepackage[colorlinks=true, linkcolor=blue]{hyperref}
\usepackage{pdfpages}
\usepackage{algorithmic}
\usepackage{algorithm}
\usepackage{listings}
\usepackage{color}
\usepackage{setspace}
%%%%%%%%%%%%%%%%%%%%%%%%%%%%%%%%%%%%%%%%%%%%%%%%%%%%%%%%
%%% SCRIPT DE TODO'S O TAREAS PENDIENTES CON BOCADILLOS.
%%%%%%%%%%%%%%%%%%%%%%%%%%%%%%%%%%%%%%%%%%%%%%%%%%%%%%%%
\usepackage{tikz}
\makeatletter \newcommand \listoftodos{\section*{Todo list} \@starttoc{tdo}}
\newcommand\l@todo[2]
    {\par\noindent \textit{#2}, \parbox{10cm}{#1}\par} \makeatother

\definecolor{orange}{rgb}{1,0.5,0}
\tikzstyle{notestyle} += [text width=\marginparwidth]
\tikzstyle{notestyleleft} += [text width=0.9\marginparwidth]
\tikzstyle{notestyle} = [draw=black, fill=orange, text width = 3cm]
\tikzstyle{notestyleleft} = [notestyle]
\tikzstyle{connectstyle} = [draw = orange, thick]
% Command for inserting a todo item
\newcommand{\todo}[1]{%
% Add to todo list
\addcontentsline{tdo}{todo}{\protect{#1}}%
%
\begin{tikzpicture}[remember picture, baseline=-0.75ex]%
    \node [coordinate] (inText) {};
\end{tikzpicture}%
%
% Make the margin par
\marginpar[%
{% Draw note in left margin
    \tikz[remember picture] \draw node[notestyleleft] (inNote) {#1};%
    \begin{tikzpicture}[remember picture, overlay]%
        \draw[connectstyle]
            ([yshift=-0.2cm] inText)
                -| ([xshift=0.2cm] inNote.east)
                -| (inNote.east);
    \end{tikzpicture}%
}%
]{% Draw note in right margin
    \tikz[remember picture] \draw node[notestyle] (inNote) {#1};%
    \begin{tikzpicture}[remember picture, overlay]%
        \draw[connectstyle]
            ([yshift=-0.2cm] inText)
                -| ([xshift=-0.2cm] inNote.west)
                -| (inNote.west);
    \end{tikzpicture}%
}%
}%

%%%%%%%%%%%%%%%%%%%%%%%%%%%%%%%%%%
%%%%%%%%%%%%%%%%%%%%%%%%%%%%%%%%%%

\floatname{algorithm}{Algoritmo}
\pagestyle{fancy}
\renewcommand{\chaptermark}[1]{\markboth{\textsc{Cap�tulo} \thechapter. #1}{}}
\renewcommand{\sectionmark}[1]{\markright{\thesection\ #1}}

\fancyhf{}
\fancyhead[LE,RO]{\thepage}
\fancyhead[LO]{\rightmark}
\fancyhead[RE]{\leftmark}

%% Dejar espacio para la regla
\renewcommand{\headrulewidth}{0.02cm} %Son 0.5pt

% Redefinici�n de 'plain' para la primera p�gina del cap�tulo
\fancypagestyle{plain}{
\fancyhf{}
\renewcommand{\headrulewidth}{0pt}
\renewcommand{\footrulewidth}{0pt}
}

% Limpiar las cabeceras de las p�ginas que quedan completamente blancas
\makeatletter
  \def\cleardoublepage{\clearpage\if@twoside \ifodd\c@page\else
  \vspace*{\fill}
    \thispagestyle{empty}
    \newpage
    \if@twocolumn\hbox{}\newpage\fi\fi\fi}
\makeatother


\setlength\headheight{20pt}


%%%%%%%%%%%%%%%%%%%%%%%%%%%%%%%%%%%%%%%%%%%%%%
%%% COMIENZA EL DOCUMENTO
%%%%%%%%%%%%%%%%%%%%%%%%%%%%%%%%%%%%%%%%%%%%%%

\pagenumbering{roman}
\makeindex
\begin{document}

\newlength{\centeroffset}
\setlength{\centeroffset}{-0.5\oddsidemargin}
\addtolength{\centeroffset}{0.5\evensidemargin}
\thispagestyle{empty}
\vspace*{\stretch{1}}
\setstretch{1.2}

\noindent
\hspace*{\centeroffset}

\renewcommand\contentsname{�ndice general}
\renewcommand\listfigurename{�ndice de Figuras}
\renewcommand\listtablename{�ndice de Tablas}
\renewcommand\bibname{Referencias}
\renewcommand\figurename{Figura}
\renewcommand\tablename{Tabla} 
\renewcommand\chaptername{Cap�tulo}
\renewcommand\appendixname{Ap�ndice}
\renewcommand\abstractname{Resumen}



%%%%%%%%%%%%%%%%%%%%%%%%%%%%%%%%%%%%%%%%%%%%%%
%%% PORTADA
%%%%%%%%%%%%%%%%%%%%%%%%%%%%%%%%%%%%%%%%%%%%%%

\thispagestyle{empty}
\vspace*{1cm}
\hrule
\mbox{ }

\begin{center}
\begin{LARGE}
Trabajo Fin de M�ster: T�tulo del Trabajo \\ [0.3em]
\end{LARGE}
\end{center}

\hrule
\vspace*{1cm}
\begin{center}
 \includegraphics[width=0.3\textwidth]{imagenes/Logo_UNED.jpg} %[width=4cm,,keepaspectratio]
\end{center}

\vfill
\begin{center}
  {\Large {\bf Trabajo Fin de M�ster}}
\end{center}

\vfill
\begin{center}
  \textbf{Nombre del Alumn@} \\ [0.2em]
  Trabajo de investigaci�n para el \\ \vspace{10pt}
  M�ster en Lenguajes y Sistemas Inform�ticos\\ \vspace{10pt}
  Universidad Nacional de Educaci�n a Distancia\\ \vspace{10pt}
  \vspace{2cm}
  Dirigido por el \\ \vspace{10pt}
  \textbf{Prof. Dr. D. Nombre Direct@r} \\ [0.5em]
  Marzo 2015\\
\end{center}


%%%%%%%%%%%%%%%%%%%%%%%%%%%%%%%%%%%%%%%%%%%%%%
%%% AGRADECIMIENTOS
%%%%%%%%%%%%%%%%%%%%%%%%%%%%%%%%%%%%%%%%%%%%%%
\chapter*{Agradecimientos}
\label{sec:Agradecimientos}

\noindent Agradecimientos si procede.


%%%%%%%%%%%%%%%%%%%%%%%%%%%%%%%%%%%%%%%%%%%%%%
%%% RESUMEN
%%%%%%%%%%%%%%%%%%%%%%%%%%%%%%%%%%%%%%%%%%%%%%
\chapter*{Resumen}
\label{sec:Resumen}
\noindent En el presente trabajo se propone una arquitectura para la detecci�n del machismo en la red de microblogging Twitter. En la actualidad, el abuso online se ha convertido en un gran problema, especialmente por la anonimidad an y interactividad de la web que facilita el incremento y permanencia de este tipo de abusos. Se trata de un campo en el que ha aumentado la producci�n cient�fica enormemente durante este mismo a�o y donde se han desarrollado competiciones con gran participaci�n por parte de la comunidad cient�fica. A lo largo del trabajo, se presenta el ciclo completo para la recolecci�n de datos, preprocesamiento y construcci�n del sistema de clasificaci�n. Se desarrolla un sistema...
Se evalua... Los resultados demuestran ...\todo{Completar cuando se avance en el trabajo}
Finalmente, se identifican algunos problemas y l�neas de trabajo futuras.


%%%%%%%%%%%%%%%%%%%%%%%%%%%%%%%%%%%%%%%%%%%%%%
%%% ABSTRACT
%%%%%%%%%%%%%%%%%%%%%%%%%%%%%%%%%%%%%%%%%%%%%%
\chapter*{Abstract}
\label{sec:Abstract}
\noindent Breve resumen del trabajo realizado y de los objetivos conseguidos en ingl�s.


%%%%%%%%%%%%%%%%%%%%%%%%%%%%%%%%%%%%%%%%%%%%%%
%%% INDICES
%%%%%%%%%%%%%%%%%%%%%%%%%%%%%%%%%%%%%%%%%%%%%%
\tableofcontents
\listoffigures
\listoftables


%%%%%%%%%%%%%%%%%%%%%%%%%%%%%%%%%%%%%%%%%%%%%%
%%% ENTRADAS A LOS FICHEROS CON LOS CAPITULOS
%%%%%%%%%%%%%%%%%%%%%%%%%%%%%%%%%%%%%%%%%%%%%%
%%%%%%%%%%%%%%%%%%%%%%%%%%%%%%%%%%%%%%%%%%%%%%
%%% INTRODUCCION
%%%%%%%%%%%%%%%%%%%%%%%%%%%%%%%%%%%%%%%%%%%%%%

\chapter{Introducci�n}
\fancyhead[RE]{\textsc{CAP�TULO} \thechapter. Introducci�n}
\label{ch:Introduccion}
\pagenumbering{arabic}

\section{Motivaci�n}
\label{sec:Motivacion}

\noindent Motivaci�n del trabajo a realizar.


El primer p�rrafo de cada secci�n no se indenta, los siguientes s�.
Las referencias a un art�culo del estado del arte se ponen as�: El trabajo  bla bla.

La presente plantilla es meramente orientativa. Tanto los cap�tulos como las secciones de cada cap�tulo son simplemente una gu�a para ayudar al alumno en el proceso de escritura del trabajo final del m�ster, pero en ning�n caso supone que los trabajos tengan que tener forzosamente esta estructura. La estructura del trabajo debe ser acordada por el alumno y su director/res, as� como adecuarse a la tem�tica abordada.

\section{Propuesta y objetivos}
\label{sec:PropuestaYObjetivos}

\noindent Que se ha realizado en el trabajo de fin de master, cuales eran los objetivos y breve resumen de los resultados obtenidos. Texto de prueba 3.

\section{Estructura del documento}
\label{sec:EstructuraDelDocumento}

\noindent Breve descripci�n de los cap�tulos del trabajo.

\begin{description}

\item \textbf{Cap�tulo 1. Introducci�n.} Este cap�tulo introduce los principales motivos que han llevado a la realizaci�n de este trabajo, as� como la problem�tica y el estado actual de la disciplina. Por �ltimo, se presentan las diferentes contribuciones del trabajo realizado.
\item \textbf{Cap�tulo 2. Estado del arte.} Este cap�tulo describe en mayor detalle la disciplina que nos ocupa, presentando su origen y su historia hasta el presente. Se muestran las t�cnicas actuales m�s utilizadas para resolver las tareas m�s relevantes del tema abordado, as� como sus debilidades.
\item \textbf{Cap�tulo 3. Sistema/M�todo/Caso de Estudio propuesto.}  En este cap�tulo se describe en profundidad el sistema/m�todo o caso de estudio propuesto.
\item \textbf{Cap�tulo 4. Evaluaci�n.} Este cap�tulo describe la metodolog�a utilizada para evaluar la propuesta realizada, a la vez que presenta los resultados obtenidos al evaluar el m�todo propuesto en diferentes tareas y sobre colecciones de evaluaci�n de distintos dominios.
\item \textbf{Cap�tulo 5. Discusi�n.} Este cap�tulo analiza y discute en profundidad los resultados obtenidos en la evaluaci�n presentada en el cap�tulo anterior.
\item \textbf{Cap�tulo 6. Conclusiones y trabajo futuro.} Este cap�tulo recopila las diferentes conclusiones extra�das del trabajo realizado, y propone algunas l�neas de trabajo futuro.
\end{description} 
%%%%%%%%%%%%%%%%%%%%%%%%%%%%%%%%%%%%%%%%%%%%%%%%%%%
%%% ESTADO DEL ARTE
%%%%%%%%%%%%%%%%%%%%%%%%%%%%%%%%%%%%%%%%%%%%%%%%%%%

\chapter{Estado del arte}
\fancyhead[RE]{\textsc{CAP\'ITULO} \thechapter. Estado del arte}
\label{ch:EstadoArte}

\noindent El presente cap�tulo tiene como objetivo presentar al lector la detecci�n del lenguaje machista en redes sociales. Para ello, se realizar� una revisi�n de los trabajos m�s relevantes en la tarea de detecci�n de lenguaje abusivo y machista, en los que se analizar�n los or�genes de esta tarea, las soluciones t�cnicas y las aportaciones m�s relevantes.

\section{Detecci�n de lenguaje o discurso del odio (\textbf{\textit{hate speech detection}}) }
\label{sec:Ejemplo_seccion}
La detecci�n del lenguaje machista o sexista est� muy relacionada con la detecci�n del lenguaje o discurso del odio en redes sociales. Existen numerosos trabajos donde se intenta detectar distintos tipos de lenguaje del odio, entre ellos el sexismo \cite{WATANABE2018,WaseemHovy2016,Georgios2018,Badjatiya2017,Zimmerman2018,Park2017,Waseem2016}. El lenguaje del odio se refiere al uso de lenguaje agresivo, violento u ofensivo hacia un grupo espec�fico de personas que comparten una propiedad en com�n, sea esta propiedad su g�nero, su raza, sus creencias o su religi�n \cite{Davidson2017}. Atendiendo a esta definici�n, se puede considerar la detecci�n del machismo como un caso particular del discurso del odio. Por ello, es muy interesante realizar una evaluaci�n de los trabajos realizados en esta l�nea de investigaci�n.

La detecci�n del lenguaje del odio es una linea de investigaci�n muy actual, el primer estudio evaluado data del a�o 2012 \cite{Xiang2012}. En este articulo se emplea un modelo de detecci�n de temas o categor�as (\textit{topic modelling}) que explota la concurrencia de palabras para la creaci�n de atributos o \textit{features} que alimentar�n un algoritmo de clasificaci�n de aprendizaje de m�quina o \textit{machine learning}. En la mayor�a de trabajos previos se empleaban soluciones basadas en patrones para la clasificaci�n de tweets. De este modo, este art�culo supone un paso muy importante hacia la automatizaci�n y a los sistemas basados en algoritmos de \textit{machine learning}. Adem�s, durante la etapa anterior a este art�culo, el uso de expresiones coloquiales y soeces en redes sociales hace dif�cil establecer las fronteras entre el uso de lenguaje ofensivo que no tiene como objetivo despreciar a ning�n grupo de personas y el lenguaje del odio \cite{Davidson2017} utilizando patrones extra�dos de la utilizaci�n del lenguaje.

Durante los �ltimos tres a�os, se han sucedido los art�culos en la tem�tica y ha aumentado considerablemente la producci�n cient�fica en este campo. En \cite{WaseemHovy2016} se aporta el primer corpus de referencia anotados que se utilizar� posteriormente en \cite{Waseem2016,Georgios2018,Badjatiya2017,Zimmerman2018,Park2017}. Est� compuesto por 16.000 \textit{tweets} etiquetados para mensajes sexistas, racistas o sin contenido ofensivo. En este primer trabajo, se sientan las bases de las soluciones aplicadas en el resto de art�culos, se utilizan atributos como los \textit{unigramas, bigramas, trigramas} y \textit{cuatri-gramas} y un algoritmo de regresi�n log�stica para la clasificaci�n.

En el art�culo desarrollado por el mismo autor \cite{Waseem2016} se propone una soluci�n similar pero se ampl�a el corpus en 4033 \textit{tweets} y se utiliza una plataforma de \textit{crowdsourcing} para anotar los mensajes. Achacan el empeoramiento de los resultados al posible sesgo que se produce en \cite{WaseemHovy2016} ya que los \textit{tweets} solo fueron etiquetados por los autores �nicamente.

En el resto de art�culos que eval�an su propuesta utilizando el corpus desarrollado por \cite{Waseem2016}, se utilizan redes neuronales para la tarea de clasificaci�n y, en algunos, en la etapa de preprocesamiento. En la soluci�n propuesta por \cite{Zimmerman2018} se aplican redes neuronales convolucionales (\textit{CNN, Convolutional Neural Network}) para codificar el texto y extraer los atributos que se utilizar�n para el clasificador final, basado tambi�n en CNNs. Esta t�cnica permite tener en cuenta la posici�n de la palabra (su contexto) para extraer los atributos de cada \textit{tweet}. Esta misma idea junto con el uso de redes neuronales recurrentes (\textit{RNN, Recurrent Neural Network}) se utiliza en \cite{Badjatiya2017} para obtener los atributos en la etapa de procesamiento. En ambos art�culos se consiguen mejorar los resultados alcanzados por \cite{Waseem2016}.

En \cite{Georgios2018} se propone un modelo basado en RNNs para abordar el problema. Adem�s se explora la idea de utilizar atributos como la tendencia al racismo o sexismo utilizando el historial de los usuarios. Se demuestra como el uso de este tipo de atributos mejora notablemente los resultados. Esta misma idea se utiliza en \cite{Chatzakouy2017} donde se detectan cuentas agresivas estudiando al usuario y su red de seguidores.

En todos los art�culos revisados anteriormente, se trata el problema como una clasificaci�n m�ltiple donde el texto se puede clasificar seg�n las etiquetas racismo, sexismo o ninguno. Sin embargo, se podr�a resolver el problema con un doble clasificador, el primero clasifica si el texto contiene lenguaje abusivo o no y el segundo realizar�a la tarea de clasificar en contenido sexista o racista \cite{Park2017}.

Un desaf�o importante en la detecci�n del lenguaje del odio en redes sociales es la separaci�n entre lenguaje ofensivo y el lenguaje que incita o promueve el odio. Davidson \cite{Davidson2017} aporta un corpus etiquetado de 25.000 \textit{tweets} para diferenciar entre estos 2 tipos de lenguaje. En su trabajo, se propone un modelo similar a \cite{Waseem2016} donde se ponen de manifiesto las dificultades de esta soluci�n para tener en cuenta el contexto de las palabras. De este modo, si se utilizan palabras que pueden expresar odio (por ejemplo, "\textit{gay}") en un contexto positivo, hay muchas probabilidades de que el sistema detecte odio en el texto. Los resultados ser�n mejorados posteriormente en \cite{WATANABE2018} donde se ampliar� el n�mero de\textit{features} y se utilizar� un algoritmo basado en �rboles de decisi�n para la tarea de clasificaci�n.











\section{Ejemplo secci�n}
\label{sec:Ejemplo_seccion}

\subsection{Ejemplo subsecci�n}
\label{sec:Ejemplo_subSeccion} 
%%%%%%%%%%%%%%%%%%%%%%%%%%%%%%%%%%%%%%%%%%%%%%%%%%%
%%% SISTEMA
%%%%%%%%%%%%%%%%%%%%%%%%%%%%%%%%%%%%%%%%%%%%%%%%%%%

\chapter{Sistema/M�todo/Caso de Estudio propuesto}
\fancyhead[RE]{\textsc{CAP�TULO} \thechapter. Sistema/M�todo/Caso de Estudio propuesto}
\label{ch:Sistema_Metodo_Caso_de_Estudio}

\noindent En este cap�tulo se describe en profundidad el sistema/m�todo o caso de estudio propuesto. La organizaci�n de este cap�tulo depender� sustancialmente del trabajo abordado.


\section{Ejemplo secci�n}
\label{sec:Ejemplo_seccion}

\subsection{Ejemplo subsecci�n}
\label{sec:Ejemplo_subSeccion} 
%%%%%%%%%%%%%%%%%%%%%%%%%%%%%%%%%%%%%%%%%%%%%%%%%%%
%%% EVALUACION
%%%%%%%%%%%%%%%%%%%%%%%%%%%%%%%%%%%%%%%%%%%%%%%%%%%

\chapter{Evaluaci�n y discusi�n}
\fancyhead[RE]{\textsc{CAP�TULO} \thechapter. Evaluaci�n}
\label{ch:Evaluacion}

\noindent En el siguiente cap�tulo se presentan los procedimientos de evaluaci�n as� como los resultados de los experimentos realizados. Adem�s, se realizar� una comparativa y discusi�n de los distintos resultados obtenidos. Para evaluar el sistema propuesto, se definen dos experimentos seg�n el procedimiento seguido para realizar la evaluaci�n del sistema. En el primero, se reservan una parte de los datos para realizar una b�squeda de los hiperpar�metros �ptimos para cada algoritmo de clasificaci�n mientras que en el segundo se emplean los par�metros por defecto para evitar el sobreajuste. Con estos experimentos, se pretende evaluar el rendimiento del sistema propuesto para detecci�n del machismo en redes sociales. Asimismo, se evaluar� el efecto que tiene en el sistema el desbalanceo existente en la clase del conjunto de datos.

\section{Metodolog�a de evaluaci�n}
\label{sec:Met_Eval}

\subsection{M�tricas de evaluaci�n}
\label{sec:Col_Eval}

\noindent Para la evaluaci�n de los resultados en clasificaci�n textual o de documentos se utiliza com�nmente la matriz de confusi�n. Se trata de una una herramienta que representa en cada columna el n�mero de predicciones de cada clase, mientras que cada fila representa a las instancias en la clase real. En la siguiente imagen se presenta un esquema de la matriz de confusi�n:

\begin{center}
	\includegraphics[width=0.5\textwidth]{imagenes/confusion_matrix_1.png} %[width=4cm,,keepaspectratio]
	\captionof{figure}{Matriz de confusi�n}	
\end{center}

Esta tabla est� formada por verdaderos positivos, verdaderos negativos, falsos positivos y falsos negativos. De este modo, si un documento es clasificado por el sistema autom�tico en la misma categor�a que la clasificaci�n manual, se considerar� como un verdadero positivo o negativo (\textit{True Positive}, TP o \textit{True Negative}, TN), mientras que si el documento es clasificado por el sistema con una categor�a diferente, se estar� ante un falso negativo o falso positivo (\textit{False Positive}, FP o \textit{False Negative}, FN).Utilizando estos cuatro componentes se calculas las medidas principales para evaluar los resultados:

\begin{itemize}
	\itemsep0em 
	\item Tasa de acierto o exactitud (\textit{accuracy}): representa el porcentaje de aciertos en relaci�n a todos los documentos clasificados.
	\[\textit{Accuracy}=\frac{TP+TN}{TP+FP+TN+FN}\]
	\item Precisi�n: representa la fracci�n de asignaciones correctas frente al total de asignaciones positivas realizadas para esa clase. Es decir, realiza una medida de la tasa de acierto para un valor de la clase.
	\[Precision=\frac{TP}{TP+FP}\]
	\item Cobertura (\textit{recall}): representa la fracci�n de asignaciones positivas respecto al conjunto real de elementos pertenecientes a la clase. Es decir, realiza una medida de la capacidad que tiene el clasificador de detectar elementos de esa clase.
	\[Cobertura=\frac{TP}{TP+FN}\]
	\item Medida-F: combina las medidas de precision y cobertura.
	\[Medida-F=\frac{2xprecisionxcobertura}{precision+cobertura}\]
\end{itemize}


\subsection{Colecci�n de evaluaci�n}
\label{sec:Col_Eval}

\noindent Las colecciones de evaluaci�n son conjuntos de datos etiquetados con informaci�n relevante para la tarea para la cual han sido desarrollados. En este caso, las colecciones de evaluaci�n para clasificaci�n de documentos est�n compuestas por textos, ya sean oraciones, p�rrafos o documentos completos, de distinta naturaleza y que est�n etiquetados con categor�as concretas. Por ejemplo, para el presente trabajo, existen tres valores posibles para esta categor�a: "MACHISTA", "NO\_MACHISTA" y "DUDOSO".

Estos conjuntos de evaluaci�n permiten intuir el rendimiento de los sistemas de clasificaci�n y compararlo con el de otros sistemas. Asimismo, en los sistemas de clasificaci�n supervisados son clave para poder ser entrenados utilizando un subconjunto de la colecci�n.

Para el presente trabajo, se utilizar� como conjunto de evaluaci�n del sistema de clasificaci�n de contenido machista el corpus presentado en el cap�tulo 4. Se trata de un corpus compuesto por 3600 \textit{tweets} recopilados mediante el uso de expresiones que pueden conllevar actitudes machistas. 

Para recuperar esta informaci�n se utilizaron los siguientes t�rminos: ``feminazi", ``loca del", ``a la cocina", zorra, ``como una ni�a", ``las feministas", ni�ata, ``como una mujer", ``en tus d�as", ``a fregar", mojigata, marimacho, nenaza, ``para ser mujer", ``odio a las mujeres", lagartona, ``A las mujeres hay que", ``las mujeres no deber�an", ``las mujeres de hoy en d�a", ``mujer al volante", ``mujer ten�as que ser", ``mucho feminismo pero", ``pareces una puta", ``para ser chica". De este modo, se recopilaron todos los mensajes escritos en la red social que contuvieran estos t�rminos durante las fechas 1/07/2018-31/12/2018.

El prop�sito principal de este corpus es la obtenci�n de texto con alto contenido machista as� como expresiones que, a�n pudiendo ser machistas, no lo sean en ese contexto. De este modo, se pretende obtener un conjunto rico en el uso de expresiones que pueden conllevar actitudes machistas en diferentes contextos. Para conseguir captar estos matices, el corpus est� etiquetado con las categor�as "MACHISTA", "NO\_MACHISTA" y "DUDOSO". 

\subsection{L�neas base (\textit{baseline})}
\label{sec:Col_Eval}

\noindent Como se ha ido introduciendo en el cap�tulo 2, las referencias en el campo de detecci�n del machismo son muy reducidas y, por tanto, es complejo encontrar alg�n trabajo comparable con el sistema desarrollado. Es por esto que en este trabajo se han desarrollado dos l�neas base con las que comparar los resultados obtenidos por el sistema dise�ado. La primera de ellas plantea un sistema de clasificaci�n basado en una regresi�n log�stica sobre los atributos \textit{tf-idf} utilizando los unigramas de cada documento. De este modo, se plantea un sistema sencillo pero pudiendo ser, en ocasiones, mucho m�s efectivo que otras aproximaciones m�s complejas que utilizan bi-gramas o categor�as gramaticales de los t�rminos, por tanto, se trata de un \textit{baseline} dif�cil de batir.

La segunda l�nea base est� basada en un clasificador sencillo que predice siempre la clase mayoritaria. En este caso, como se puede observar en la tabla 4.7, la clase mayoritaria ser�a "NO\_MACHISTA" y, por tanto, este sistema clasificar�a todos los registros de entrada con este valor de clase. La intenci�n de esta l�nea base es comparar los resultados del sistema con otro hipot�tico no informado que no es capaz de "aprender" ning�n patr�n del conjunto de datos de entrenamiento.

\subsection{Experimento 1: B�squeda de hiperpar�metros mediante la optimizaci�n de la medida F1}
\label{sec:Col_Eval}

\noindent El primer experimento realizado trata de configurar cada algoritmo de clasificaci�n para la tarea especifica que van a desarrollar. Como se introdujo en el cap�tulo 5 para la tarea de clasificaci�n se emplean 3 algoritmos distintos disponibles en ``scikit-learn'': Regresi�n log�stica, Random Forest y SVM. En este primer experimento, se realiza una b�squeda para los siguientes par�metros:

\begin{itemize}
	\itemsep0em 
	\item Regresi�n log�stica: C = [1, 10], class\_weight' = [None, 'balanced'].
	\item Random Forest: n\_estimators" = [250, 450], bootstrap' = (True, False), max\_depth'= [None, 30].
	\item SVM: C = [1, 10, 100, 10000], gamma = [0.001, 0.1, 0.6, 'auto'], kernel = 'rbf'.
\end{itemize}

Para ello, se sigue el procedimiento presentado en la figura 6.2 de forma iterativa. En el primer paso, se realizan diez reparticiones aleatorias en dos conjuntos de datos: entrenamiento (training) y testeo (testing). Para el conjunto de training, se reservan el 30\% de los datos y para el test, el resto. Para cada uno de los diez repartos, se utiliza el conjunto de training para la b�squeda de hiperpar�metros y el testing para evaluar los resultados con los par�metros encontrados.

Para la b�squeda de par�metros, se realiza una validaci�n cruzada (\textit{cross validation}) con cinco grupos. En este proceso, se realizan cinco conjuntos (que coincidir�an con los 5 "splits" de la figura) con los datos de entrenamiento y, posteriormente, se realiza el entrenamiento en cuatro de ellos y el testeo en el grupo restante. Este proceso permite obtener los par�metros que mejor han funcionado en el proceso seg�n el valor de la medida F1. Este proceso se repetir�a para cada uno de los 5 grupos.

En la siguiente etapa del proceso se utilizar�a el segundo conjunto de datos reservado para el testeo que coincidir�a con el 70\% de todo el corpus para realizar la evaluaci�n final. Utilizando los par�metros de entrada obtenidos en la etapa anterior, se realizar�a una validaci�n cruzada con 10 grupos del conjunto de testing. De nuevo, en este proceso se realizar�a un reparto en diez grupos, donde nueve de ellos ser�n utilizados para el entrenamiento y el grupo restante para el testeo, repitiendo el proceso diez veces, una por grupo. Estas dos etapas, se repiten para los diez repartos indicados al inicio.

Este experimento permite medir el resultado de un sistema dise�ado especificamente para esta tarea pues la configuraci�n de los algoritmos de clasificaci�n se realiza seg�n los datos del corpus. Adem�s, al realizar diez iteraciones para el proceso, la varianza de los resultados se reduce y son menos dependientes del tipo de datos con el que se ha entrenado. La desventaja principal de este m�todo es que es necesario reservar un conjunto de datos para la b�squeda de par�metros y se reduce la informaci�n de la que dispondr� el sistema de clasificaci�n definitivo que realizar� la predicci�n.

\begin{center}
	\includegraphics[scale=0.4,keepaspectratio]{imagenes/grid_search_cross_validation.png} %[width=4cm,,keepaspectratio]
	\captionof{figure}{B�squeda de hiperpar�metros mediante la optimizaci�n de la medida F1}
\end{center}

\subsection{Experimento 2: Cross validation con par�metros por defecto}
\label{sec:Col_Eval}

\noindent El segundo experimento consiste en una �nica validaci�n para todo el corpus utilizando los par�metros por defecto para todos los algoritmos de clasificaci�n utilizados.

En este caso, se ha optado por una validaci�n cruzada con diez grupos. En la figura 6.3 se presenta un ejemplo equivalente para cinco grupos. En este procedimiento, se realizar�a una divisi�n del conjunto en 10 grupos del mismo tama�o del corpus y, de forma iterativa, se utilizar�n nueve de ellos para el entrenamiento y el grupo restante para el testeo.

Este m�todo permite evaluar un sistema m�s general cuyos par�metros de configuraci�n no est�n dise�ados para los datos de entrenamiento de los que se disponen. Esto permite mejorar la capacidad de generalizaci�n del sistema y evitar un posible sobreajuste.

\begin{center}
	\includegraphics[scale=0.55,keepaspectratio]{imagenes/cross_validation_5.png} %[width=4cm,,keepaspectratio]
	\captionof{figure}{Validaci�n cruzada k=5}
\end{center}

\section{Resultados experimento 1}
\label{sec:Metric_Eval}

La tabla 6.1 muestra los resultados promedio de exactitud (\textit{Accuracy}), medida F1, cobertura (\textit{Recall}) y precisi�n. Con el m�todo de evaluaci�n descrito para el primer experimento, el algoritmo \textit{Random Forest} alcanza una mayor tasa de acierto y precisi�n mientras que la regresi�n log�stica alcanza los mejores resultados para la medida F1 y \textit{recall}. 

En relaci�n a la comparaci�n de los resultados obtenidos por el m�todo con las dos l�neas bases, en este caso las diferencias con respecto a la aproximaci�n basada en unigramas son de unos cuatro puntos porcentuales para cada medida. Por tanto, el sistema mejora esta primera aproximaci�n en todas las medidas pero, como se preve�a, la l�nea base ya es un buen punto de partida del sistema.

En el caso de la l�nea base basada en el clasificador sencillo de la clase mayoritaria, si se pueden observar grandes diferencias en las m�tricas de calidad. Esto indica, que cualquiera de las soluciones propuestas ser� mucho m�s adecuada que un clasificador basado en una �nica regla sencilla.

% Please add the following required packages to your document preamble:
% \usepackage{booktabs}
\begin{table}[]
	\centering
	\begin{tabular}{@{}lllll@{}}
		\toprule
		& \textbf{Accuracy} & \textbf{F1} & \textbf{Recall} & \textbf{Precision} \\ \midrule
		\textbf{Baseline (tf-idf)} & 0.68 & 0.59 & 0.62 & 0.59 \\
		\textbf{Baseline} & 0.61 & 0.2 & 0.3 & 0.24 \\
		\textbf{LR} & 0.7 & \textbf{0.62} & \textbf{0.64} & 0.62 \\
		\textbf{RF} & \textbf{0.72} & 0.6 & 0.57 & \textbf{0.67} \\
		\textbf{SVM} & 0.7 & 0.61 & 0.63 & 0.61 \\ \bottomrule
	\end{tabular}
	\caption{Resultados experimento 1}
	\label{tab:my-table}
\end{table}


\section{Resultados experimento 2}
\label{sec:Col_Eval}

La tabla 6.2 muestra los resultados promedio de exactitud (\textit{Accuracy}), medida F1, cobertura (\textit{Recall}) y precisi�n. Para este segundo experimento, el algoritmo RF consigue de nuevo la mejor tasa de acierto y precisi�n, mientras que en este caso es el algoritmo SVM el que obtiene los mejores resultados en cuanto a la medida F1 y el \textit{recall}. Sin embargo, es importante destacar que el comportamiento de los tres clasificadores es bastante similar en l�neas generales.

En relaci�n con el m�todo de evaluaci�n propuesto en el experimento 1, se puede observar una peque�a mejorar�a de un punto para cada medida de calidad.

En relaci�n a la comparaci�n de los resultados obtenidos por el m�todo con las dos l�neas bases, de nuevo, las diferencias con respecto a la aproximaci�n basada en unigramas son de unos cuatro puntos porcentuales para cada medida. 

En el caso de la l�nea base basada en el clasificador sencillo de la clase mayoritaria, se acrecientan las diferencias pues este segundo m�todo de evaluaci�n parece mejorar, en l�neas generales, los resultados del primer m�todo.

% Please add the following required packages to your document preamble:
% \usepackage{booktabs}
\begin{table}[]
	\centering
	\begin{tabular}{@{}lllll@{}}
		\toprule
		& \textbf{Accuracy} & \textbf{F1} & \textbf{Recall} & \textbf{Precision} \\ \midrule
		\textbf{Baseline (tf-idf)} & 0.69 & 0.58 & 0.56 & 0.62 \\
		\textbf{Baseline} & 0.61 & 0.2 & 0.3 & 0.24 \\
		\textbf{LR} & 0.72 & \textbf{0.63} & 0.62 & 0.65 \\
		\textbf{RF} & \textbf{0.73} & 0.61 & 0.58 & \textbf{0.68} \\
		\textbf{SVM} & 0.72 & \textbf{0.63} & \textbf{0.64} & 0.63 \\ \bottomrule
	\end{tabular}
	\caption{Resultados experimento 2}
	\label{tab:my-table}
\end{table}

\section{Efecto del desbalanceo de la clase}
\label{sec:Resultados}




%%%%%%%%%%%%%%%%%%%%%%%%%%%%%%%%%%%%%%%%%%%%%%%%%%%
%%% DISCUSION
%%%%%%%%%%%%%%%%%%%%%%%%%%%%%%%%%%%%%%%%%%%%%%%%%%%

\chapter{Discusi�n}
\fancyhead[RE]{\textsc{CAP�TULO} \thechapter. Discusi�n}
\label{ch:Discusion}

\noindent Este cap�tulo analiza y discute en profundidad los resultados obtenidos en la evaluaci�n presentada en el cap�tulo anterior. La estructura de este cap�tulo depender� del tema del trabajo y de la estructura del cap�tulo anterior.

\section{Ejemplo secci�n}
\label{sec:Ejemplo_seccion}

\subsection{Ejemplo subsecci�n}
\label{sec:Ejemplo_subSeccion} 
%%%%%%%%%%%%%%%%%%%%%%%%%%%%%%%%%%%%%%%%%%%%%%%%%%%
%%% CONCLUSIONES Y TRABAJO FUTURO
%%%%%%%%%%%%%%%%%%%%%%%%%%%%%%%%%%%%%%%%%%%%%%%%%%%

\chapter{Conclusiones y trabajo futuro}
\fancyhead[RE]{\textsc{CAP�TULO} \thechapter. Conclusiones y trabajo futuro}
\label{ch:Conclusiones y trabajo futuro}

\noindent Este cap�tulo recopila las diferentes conclusiones extra�das del trabajo realizado, y propone algunas l�neas de trabajo futuro. Las siguientes secciones son las que suelen contener este tipo de cap�tulos, aunque pueden variar dependiendo del tema del trabajo.

\section{Conclusiones}
\label{sec:Conclu}


\section{Trabajo futuro}
\label{sec:Trab_Fut} 

%%%%%%%%%%%%%%%%%%%%%%%%%%%%%%%%%%%%%%%%%%%%%%%%%%%
%%% BIBLIOGRAF�A
%%%%%%%%%%%%%%%%%%%%%%%%%%%%%%%%%%%%%%%%%%%%%%%%%%%
\bibliographystyle{fullname_esp}
\cleardoublepage
\addcontentsline{toc}{chapter}{Bibliograf�a}
\chapter*{Bibliograf�a}
\fancyhead[RE]{BIBLIOGRAF�A}
\fancyhead[LO]{BIBLIOGRAF�A}
\bibliography{Bibliografia}


\newpage
\newpage

%%%%%%%%%%%%%%%%%%%%%%%%%%%%%%%%%%%%%%%%%%%%%%%%%%%
%%% APENDICES SI PROCEDE (COMENTAR)
%%%%%%%%%%%%%%%%%%%%%%%%%%%%%%%%%%%%%%%%%%%%%%%%%%%
\appendix
\chapter{Publicaciones}
\label{ch:publicaciones}
\fancyhead[RE]{\textsc{AP�NDICE} \thechapter. Publicaciones}
\fancyhead[LO]{\textsc{AP�NDICE} \thechapter. Publicaciones}

Publicaciones derivadas del trabajo realizado.


%%%%%%%%%%%%%%%%%%%%%%%%%%%%%%%%%%%%%%%%%%%%%%%%%%%
%%% lISTA DE TAREAS PENDIENTES.
%%%%%%%%%%%%%%%%%%%%%%%%%%%%%%%%%%%%%%%%%%%%%%%%%%%
\newpage
\newpage
%Eliminar si no hay todo's
\listoftodos


\end{document}

%%%%%%%%%%%%%%%%%%%%%%%%%%%%%%%%%%%%%%%%%%%%%%%%%%%
%%% SCRIPT FINAL DE TODO'S O TAREAS PENDIENTES.
%%%%%%%%%%%%%%%%%%%%%%%%%%%%%%%%%%%%%%%%%%%%%%%%%%%
% Make the margin par
\marginpar{%
    \begin{tikzpicture}[remember picture]%
        \draw node[notestyle] (inNote)
                 {#1};%
    \end{tikzpicture}%
}%
%
\begin{tikzpicture}[remember picture, overlay]%
    \draw[connectstyle]
        ([yshift=-0.2cm] inText)
            -| ([xshift=-0.2cm] inNote.west)
            -| (inNote.west);
\end{tikzpicture}%

}% 